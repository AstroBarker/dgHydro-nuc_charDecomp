3d 3d 3d
put Astronum Eq 15 here as well, restate U = (...)
For the non-relativistic Euler equations, the state and flux vectors are given by
\begin{equation}
  \vect{U}=\big(\rho,\rho v_{j},E\big)^{T},~
  \vect{F}^{i}=\big(\rho v^{i},\Pi^{i}_{~j},(E+p)v^{i}\big)^{T},~
\end{equation}
where $\rho$ and $v^{i}$ are the mass density and components of the fluid three-velocity, respectively.
The stress tensor is $\Pi^{i}_{~j}=\rho u^{i} u_{j}+p\delta^{i}_{~j}$, and $E=e+\f{1}{2}\rho u_{i} u^{i}$ is the fluid (internal plus kinetic) energy density.

%\begin{footnotesize}
	\begin{align}
	  \centering
		\pderiv{\mathbf{F}^{1}(\mathbf{U})}{\mathbf{U}}
		= \left[
			\begin{array}{cccccc}
				0 & 1 & 0 & 0 & 0 & 0 \\
				-v_{1}^{2} -P_{\tau}\tau^{2} - P_{\epsilon}\tau(\epsilon - \frac{v^{2}}{2}) & v_{1}(2-P_{\epsilon}\tau)  & -P_{\epsilon}v_{2}\tau & -P_{\epsilon}v_{3}\tau  & P_{\epsilon}\tau  & P_{D_{e}} \\
				-v_{1}v_{2} & v_2 & v_1 & 0 & 0 & 0 \\
				-v_{1}v_{3} & v_{3} & 0 & v_{1} & 0 & 0 \\
				v_{1}(-H - P_{\tau}\tau^{2} -P_{\epsilon}\tau(\epsilon - \frac{v^{2}}{2})) & H - P_{\epsilon}v_{1}^{2}\tau  & -P_{\epsilon}v_{1}v_{2}\tau & -P_{\epsilon}v_{1}v_{3}\tau  & v_{1}(1+P_{\epsilon}\tau) & v_{1} P_{D_{e}} \\
				-v_{1} y_{e} & y_{e} & 0 & 0 & 0 & v_{1} \\
			\end{array}
	    \right]
	\end{align}

	\begin{align}
		\centering
		\pderiv{\mathbf{F}^{2}(\mathbf{U})}{\mathbf{U}}
		= \left[
			\begin{array}{cccccc}
				0 & 0 & 1 & 0 & 0 & 0 \\
        -v_{1}v_{2} & v_{2} & v_{1} & 0 & 0 & 0 \\
				-v_{2}^{2} -P_{\tau}\tau^{2} - P_{\epsilon}\tau(\epsilon - \frac{v^{2}}{2}) & -P_{\epsilon}v_{1}\tau &
				  v_{2}(2-P_{\epsilon}\tau) &  -P_{\epsilon}v_{3}\tau & P_{\epsilon}\tau & P_{D_{e}} \\
				-v_{2}v_{3} & 0 & v_{3} & v_{2} & 0 & 0 \\
				v_{2}(-H - P_{\tau}\tau^{2} -P_{\epsilon}\tau(\epsilon - \frac{v^{2}}{2})) &
				  - P_{\epsilon}v_{1}v_{2}\tau & H - P_{\epsilon}v_{2}^{2}\tau &
				  - P_{\epsilon}v_{2}v_{3}\tau & v_{2}(1+P_{\epsilon}\tau) & v_{2}P_{D_{e}} \\
				-v_{2}y_{e} & 0 & y_{e} & 0 & 0 & v_{2}\\
			\end{array}
			\right]
	\end{align}

	\begin{align}
		\centering
		\pderiv{\mathbf{F}^{3}(\mathbf{U})}{\mathbf{U}}
		= \left[
			\begin{array}{cccccc}
				0 & 0 & 0 & 1 & 0 & 0 \\
        -v_{1}v_{3} & v_{3} & 0 & v_{1} & 0 & 0 \\
				-v_{2}v_{3} & 0 & v_{3} & v_{2} & 0 & 0 \\
				-v_{3}^{2} -P_{\tau}\tau^{2} - P_{\epsilon}\tau(\epsilon - \frac{v^{2}}{2}) & -P_{\epsilon}v_{1}\tau &
				  -P_{\epsilon}v_{2}\tau &  v_{3}(2 - P_{\epsilon}\tau) & P_{\epsilon}\tau & P_{D_{e}} \\
				v_{3}(-H - P_{\tau}\tau^{2} -P_{\epsilon}\tau(\epsilon - \frac{v^{2}}{2})) &
				 - P_{\epsilon}v_{1}v_{3}\tau & -P_{\epsilon}v_{2}v_{3}\tau & H - P_{\epsilon}v_{3}^{2}\tau &
				  v_{3}(1+P_{\epsilon}\tau) & v_{3}P_{D_{e}} \\
				-v_{3}y_{e} & 0 & y_{e} & 0 & 0 & v_{3}\\
			\end{array}
			\right]
	\end{align}

The eigenvalues of the flux Jacobian are given by the diagonal matrix

\begin{align}
\Lambda_{i} =
\begin{bmatrix}
  v_{i} - c_{s} & 0 & 0& 0& 0& 0 \\
  0 & v_{i} & 0 & 0 & 0 & 0      \\
  0 & 0 & v_{i} & 0 & 0 & 0      \\
  0 & 0 & 0 & v_{i} & 0 & 0      \\
  0 & 0 & 0 & 0 & v_{i} & 0      \\
  0 & 0 & 0 & 0 & 0 & v_{i} + c_{s}
\end{bmatrix}
\end{align}
where $c_{s} = \sqrt{\Gamma P \tau}$, with
$\Gamma = \left(\tau (P P_{\epsilon} - P_{\tau}) + P_{D_e} y_{e} \tau^{-1}\right) P^{-1}$, is
the local sound speed.

The right eigenvectors are then given by the column vectors of the following matrices
\begin{align*}
  \mathcal{R}_{1} =
  \left[
  \begin{array}{cccccc}
    1 & 0 & 1 & 1 & 0 & 1 \\
    v_{1}-c & 0 & v_{1} & v_{1} & 0 & c+v_{1} \\
    v_{2} & 1 & 0 & 0 & 0 & v_{2} \\
    v_{3} & 0 & 0 & 0 & 1 & v_{3} \\
    h-c v_{1} & v_{2} & \beta_{1} & 0 & v_{3} & h+c v_{1} \\
    y_{e}  & 0 & 0 & \frac{\tau  \chi_{1} }{2 P_{D_{e}}} & 0 & y_{e}  \\
  \end{array}
  \right]
\end{align*}

\begin{align*}
  \mathcal{R}_{2} =
  \left[
  \begin{array}{cccccc}
    1 & 0 & 1 & 1 & 0 & 1 \\
    v_{1} & 1 & 0 & 0 & 0 & v_{1} \\
    v_{2}-c & 0 & v_{2} & v_{2} & 0 & c+v_{2} \\
    v_{3} & 0 & 0 & 0 & 1 & v_{3} \\
    h-c v_{2} & v_{1} & \beta_{2} & 0 & v_{3} & h+c v_{2} \\
    y_{e}  & 0 & 0 & \frac{\tau  \chi_{2} }{2 P_{D_{e}}} & 0 & y_{e}  \\
  \end{array}
  \right]
\end{align*}


where the following definitions have been used:
$h_{n} = \frac{c^2}{P_{\epsilon}\tau} + k $, $k = \frac{-y_{e} P_{D_{e}} \tau^{-1}
+ P_{\epsilon} (\frac{1}{2}v^2 + \epsilon) + P_{\tau}\tau}{P_{\epsilon}}$,
$\delta_{1} = v_{1}^{2}-v_{2}^{2}-v_{3}^{2}$,
$\chi_{1} = P_{\epsilon} ( \delta_{1} + 2\epsilon) + 2P_{\tau}\tau$, and
$\beta_{1} = \frac{1}{2} (\delta_{1}+2 \epsilon +\frac{2 P_{\tau} \tau }{P_{\epsilon}})$.



The left eigenvectors are given by the row vectors of the inverse matrix $\mathcal{L}_{1} = \mathcal{R}_{1}^{-1}$

\begin{align*}
  \mathcal{R}_{1}^{-1} = \frac{1}{c^2}
  \left[
  \begin{array}{cccccc}
   \frac{1}{4} (2 c  v_{1}+\omega ) & \frac{1}{2} (-c- \phi_{1} ) & -\frac{1}{2} \phi_{2}
    & -\frac{1}{2} \phi_{3}  & \frac{P_{\epsilon} \tau }{2} & \frac{P_{D_{e}}}{2}
     \\
   -\frac{v_{2} \omega }{2} & \phi_{1} v_{2}  & c^2+\phi_{2} v_{2}  &
     \phi_{3} v_{2}  & -\phi_{2}  & -P_{D_{e}} v_{2}
     \\
   \frac{2 \chi_{1}  c^2+\alpha_{1}  \omega \tau^{-1} }{2 \chi_{1} } & -\frac{\phi_{1} \alpha_{1}  }{\chi_{1} \tau } &
     -\frac{\phi_{2} \alpha_{1}  }{\chi_{1} \tau } & -\frac{\phi_{3} \alpha_{1} }{\chi_{1} \tau } &
     \frac{P_{\epsilon} \alpha_{1} }{\chi_{1} } & \frac{P_{D_{e}} \left(\alpha_{1} -2 c^2\right)}{\tau \chi_{1} }
      \\
   -\frac{y_{e} P_{D_{e}} \omega }{\chi_{1} \tau } & \frac{2 y_{e} P_{D_{e}} \phi_{1} }{\chi_{1} \tau } & \frac{2 y_{e} P_{D_{e}}
     \phi_{2} }{\chi_{1} \tau} & \frac{2 y_{e} P_{D_{e}} \phi_{3} }{\chi_{1} \tau} & -\frac{2 y_{e}
     P_{D_{e}} P_{\epsilon} }{\chi_{1} } & \frac{2 P_{D_{e}} \left(c^2-y_{e} P_{D_{e}} \right)}{\tau \chi_{1} }
      \\
   -\frac{v_{3} \omega }{2} & \phi_{1} v_{3}  & \phi_{2} v_{3}   &
     c^2+\phi_{3} v_{3}  & -\phi_{3}  & -P_{D_{e}} v_{3}
      \\
   \frac{1}{4} (\omega -2 c  v_{1}) & \frac{1}{2} (c-\phi_{1} ) & -\frac{1}{2} \phi_{2}
       & -\frac{1}{2} \phi_{3}  & \frac{P_{\epsilon} \tau }{2} & \frac{P_{D_{e}}}{2}
     \\
  \end{array}
  \right]
\end{align*}

\begin{align*}
  \mathcal{R}_{2}^{-1} = \frac{1}{c^2}
  \left[
  \begin{array}{cccccc}
   \frac{1}{4} (2 c  v_{2}+\omega ) & -\frac{1}{2} \phi_{1} & -\frac{1}{2} (c+\phi_{2})
    & -\frac{1}{2} \phi_{3}  & \frac{P_{\epsilon} \tau }{2} & \frac{P_{D_{e}}}{2}
     \\
   -\frac{v_{2} \omega }{2} & c^2 + \phi_{1} v_{1}  & \phi_{2} v_{1}  &
     \phi_{3} v_{1}  & -\phi_{1}  & -P_{D_{e}} v_{1}
     \\
   \frac{2 \chi_{2}  c^2+\alpha_{2}  \omega \tau^{-1} }{2 \chi_{2} } & -\frac{\phi_{1} \alpha_{2}  }{\chi_{2} \tau } &
     -\frac{\phi_{2} \alpha_{2}  }{\chi_{2} \tau } & -\frac{\phi_{3} \alpha_{2} }{\chi_{2} \tau } &
     \frac{P_{\epsilon} \alpha_{2} }{\chi_{2} } & \frac{P_{D_{e}} \left(\alpha_{2} -2 c^2\right)}{\tau \chi_{2} }
      \\
   -\frac{y_{e} P_{D_{e}} \omega }{\chi_{2} \tau } & \frac{2 y_{e} P_{D_{e}} \phi_{1} }{\chi_{2} \tau } & \frac{2 y_{e} P_{D_{e}}
     \phi_{2} }{\chi_{2} \tau} & \frac{2 y_{e} P_{D_{e}} \phi_{3} }{\chi_{2} \tau} & -\frac{2 y_{e}
     P_{D_{e}} P_{\epsilon} }{\chi_{2} } & \frac{2 P_{D_{e}} \left(c^2-y_{e} P_{D_{e}} \right)}{\tau \chi_{2} }
      \\
   -\frac{v_{3} \omega }{2} & \phi_{1} v_{3}  & \phi_{2} v_{3}   &
     c^2+\phi_{3} v_{3}  & -\phi_{3}  & -P_{D_{e}} v_{3}
      \\
   \frac{1}{4} (\omega -2 c  v_{2}) & \frac{1}{2} \phi_{1}  & \frac{1}{2} (c-\phi_{2})
       & -\frac{1}{2} \phi_{3}  & \frac{P_{\epsilon} \tau }{2} & \frac{P_{D_{e}}}{2}
     \\
  \end{array}
  \right]
\end{align*}
where $\phi_{i} = P_{\epsilon}\,\tau\, v_{i}$,
$\omega = \tau\, (P_{\epsilon}\,(v^2 - 2\epsilon) - 2\,P_{\tau}\,\tau)$, and
$\alpha_{i} = 2 y_{e} P_{D_{e}} - \tau \chi_{i}$.

%\end{footnotesize}
