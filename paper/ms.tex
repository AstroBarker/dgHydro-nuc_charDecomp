\documentclass[onecolumn]{aastex62}
\usepackage{times}
\usepackage{epsfig}
\usepackage{amsmath, amsthm, amssymb}
\usepackage{color}
\usepackage{microtype}
\usepackage{float}
\usepackage{stfloats}
\usepackage{natbib}
\usepackage{verbatim}
%\usepackage{placeins}
\graphicspath{{./figures/}}
\usepackage{wrapfig}
\hypersetup{backref,breaklinks,colorlinks,citecolor=blue}
\usepackage[all]{hypcap}
\usepackage{definitions}
\usepackage{subfiles}

\usepackage{placeins}

\bibliographystyle{apj}

\newcommand{\subdate}{\today}
\newcommand{\shortauth}{Barker et al.}
\newcommand{\slugcom}{Draft - \today}
\newcommand{\TVD}{\textnormal{\tiny\textsc{Tvd}}}
\newcommand{\TCI}{\textnormal{\tiny\textsc{TCI}}}

%\newcommand{\slugcom}{Submitted to ApJL on \subdate}
%\slugcomment{\slugcom}


\lefthead{\sc \footnotesize \slugcom \hfill \shortauth}
\righthead{\sc \footnotesize \slugcom \hfill \shortauth}
%}

\bibliographystyle{aasjournal}
\received{\today}
\shorttitle{Discontinuous Galerkin methods with a nuclear EOS}

\begin{document}

\title{\thornado-hydro: Generalizing discontinuous Galerkin methods for a nuclear
equation of state for supernova hydrodynamics}
\author{Brandon Barker}
\affiliation{Department of Physics and Astronomy, University of Tennessee Knoxville, TN 37996}

\author{Eirik Endeve}
\affiliation{Department of Physics and Astronomy, University of Tennessee Knoxville, TN 37996}
\affiliation{Computer Science and Mathematics Division, Oak Ridge National Laboratory, TN 37831}
\affiliation{Joint Institute for Computational Sciences, Oak Ridge National Laboratory, TN 37831}

\author{Anthony Mezzacappa}
\affiliation{Department of Physics and Astronomy, University of Tennessee Knoxville, TN 37996}
\affiliation{Joint Institute for Computational Sciences, Oak Ridge National Laboratory, TN 37831}


\begin{abstract}
A problem of high importance in computational astrophysics is obtaining accurate
solutions to the Euler equations of hydrodynamics. We are interested in solving
the Euler equations in the context of core collapse supernovae.
The toolkit for high-order neutrino-radiation hydrodynamics (\thornado)
is being developed for core-collapse supernova (CCSN) simulations and related problems utilizing a spatial discretization
based on the discontinuous Galerkin (DG) method.
The Euler equations form a hyperbolic set of partial differential equations.
In the quasi-linear form, the system can be represented
as a set of independent advection equations that can be limited separately.
This use of characteristic variables also increases the efficiency of the slope
limiting process for high-order DG methods. However, the addition of a
tabulated nuclear equation of state to the Euler equations makes the decomposition nontrivial.
We introduce the framework for the characteristic decomposition of the Euler equations
with the inclusion of the nuclear EOS terms and
present results from some initial tests using a third-order scheme.
The results confirm that performing limiting
on the characteristic variables provides better numerical solutions and is perhaps
suited to applications in CCSN simulations.
\end{abstract}

\keywords{supernovae: general -- hydrodynamics -- equation of state -- methods: numerical -- discontinuous Galerkin}


\section{Introduction}
\label{sec:Intro}
The core collapse supernovae (CCSNe) explosion mechanism is fundamentally three-dimensional
in nature \citep[see e.g.,][]{blondin:2006, muller:2012, oconnor:2018b}. Alongside general
reletavistic gravity, complex nuclear matter equations of state (EOS), and
neutrino transport, hydrodynamics must be accurately modelled,
creating a very challenging and compelling problem. Hydrodynamic instabilities are critical
in aiding the explosion, with complex phenomena such as turbulence and
convection playing key roles in the CCSN mechanism
\citep{murphy:2011, murphy:2013, couch:2013, couch:2015a, radice:2016, mabanta:2018}.
The accurate and efficient modelling of the supernova hydrodynamics is imperative if the
explosion is to be realistically modelled.
For in-depth reviews of the CCSN mechanism, see
\citet{bethe:1990, janka:2007, janka:2012a, janka:2016, burrows:2013, hix:2014, muller:2016, couch:2017}.

Discontinuous Galerkin (DG) methods have been shown by others
\citep[see e.g.,][]{radice:2011, schaal:2015a, zanotti:2015, dumbser:2018} to have a
high potential for applications to astrophysical problems. They are a desireable
choice for modelling fluid flows in CCSNe. These methods combine elements of
spectral and finite volume methods, and achieve high-order accuracy
on a compact stencil.

The {\bf t}oolkit for {\bf h}igh-{\bf or}der {\bf n}eutrino-r{\bf ad}iation hydr{\bf o}dynamics
(\thornado)\footnote{\url{https://github.com/endeve/thornado}}is being developed to simulate neutrino-radiation
hydrodynamics in CCSNe and related applications in nuclear astrophysics.
The spatial discretization of solvers for hyperbolic partial differential equations in \thornado\, is based on the DG method.
Whether the high-order approach will improve accuracy and efficiency of CCSN models remains to be demonstrated.
In this paper we provide an initial description and encouraging demonstration
of the DG method implemented in \thornado\, to solve the non-relativistic Euler equations.
We focus on basic one-dimensional tests focusing on the implementation of limiters in \thornado.

The DG slope limiting techniques implemented in \thornado\, are similar to those
used by \cite{schaal:2015a}, but is supplemented with a troubled-cell
indicator \citep{fu:2017} to prevent excessive limiting (e.g., around smooth extrema).
It has been shown \citep[see e.g.,][]{schaal:2015a, cockburn:1998, cockburn:1989}
that limiting on the chracteristic variables (characteristic limiting),
which represent advected quantities, shows an improvement over limiting
directly on the conserved variables (componentwise limiting).
In this paper, we develop a chracteristic
limiter for use with a general tabulated nuclear matter EOS in \thornado.
In the case of the ideal EOS, the eigenvectors and eigenvalues which diagonalize
the flux Jacobian, which are necessary to transform to the characteristic variables,
are known analytically. All of the thermodynamic derivatives have analytic forms.
The inclusion of a tabulated nuclear EOS complicates matters, as derivatives
must be appriximated by interpolating from the EOS table and the diagonalization
becomes non-trovial.
We provide explicit analytic expressions for the matrices which diagonalize the
flux Jacobian of the Euler equations.

This paper is organized as follows: in Section~\ref{sec:eulereq} we briefly
describe the Euler equations of gas dynamics including a conservation law
appropriate for a nuclear EOS. Section~\ref{sec:DG} describes our
numerical implementation of the DG method in the one dimensional (1D) case.
Section~\ref{sec:eigen} presents the eigenvectors used in the diagonalization
of the flux Jacobian of the Euler equations, and Section~\ref{sec:results}
shows the results of preliminary 1D hydrodynamic Riemann problem similar to that
of Sod \citep{sod:1978}.

\section{Euler Equations of Gas Dynamics in Cartesian Coordinates}
\label{sec:eulereq}
The non-relativistic Euler equations of gas dynamics
\citep[see, e.g.,][for details in the case of an ideal EOS]{leveque:2002}
in the absence of sources with a nuclear matter EOS are given by
the equations of conservation of mass,
\beq
  \partial_{t} \rho + \divergence{} (\rho\,  \mathbf{v}) = 0
  \label{eq:massConservation}
\eeq
conservation of momentum,
\beq
%	\partial_{t}(\rho \mathbf{v}) + \divergence{}(\rho u^2 + p) = 0
  \partial_{t}(\rho\,  \mathbf{v}) + \divergence{}(\rho\,  \mathbf{v} \otimes \mathbf{v} + P\, \mathbf{I}) = 0
  \label{eq:momentumConservation}
\eeq
conservation of energy,
\beq
  \partial_{t} E + \divergence{}\left[(E+P)\, \mathbf{v}\right] = 0
  \label{eq:energyConservation}
\eeq
and conservation of electrons
\beq
  \partial_{t} D_{e} + \divergence{}(D_{e}\, \mathbf{v}) = 0
  \label{eq:electronConservation}
\eeq
where $\rho$ represents mass density, $\mathbf{v}$ the fluid velocity vector,
$P$ the fluid pressure, $D_{e}=\rho y_e$ where $y_e$ is the electron fraction,
$E=\epsilon \rho +\frac{1}{2}\rho v^2$ the total energy (internal plus kinetic),
$\epsilon$ is the specific internal energy, and $\mathbf{I}$ is the identity tensor.
\eqref{eq:massConservation}-\eqref{eq:electronConservation}. The inclusion of
Equation \eqref{eq:electronConservation} is because we require a nuclear EOS.
These equations are closed by a tabulated EOS where the pressure is given by a function of
density, temperature $T$ and the electron fraction: $P = P(\rho, T, y_e)$.
We may rewrite Equations \eqref{eq:massConservation}-\eqref{eq:electronConservation}
in a more convenient way:

\beq
  \partial_{t}\mathbf{U} + \divergence{}\mathbf{F}(\mathbf{U}) = 0,
  \label{eq:conservation}
\eeq
where $\mathbf{U} =(\rho,\rho \mathbf{v},E, D_{e})^{T}$ is the vector of conserved quantities
and $\mathbf{F}(\mathbf{U})=(\rho \mathbf{v},
\rho\, \mathbf{v} \otimes \mathbf{v} + P\, \mathbf{I},(E+P)\mathbf{v}, D_{e}\mathbf{v})^{T}$
is the flux vector.

\section{Numerical Implementation}
\label{sec:DG}

\subsection{The Discontinuous Galerkin Methods}
In our solver we have chosen the discontinuous Galerkin (DG) method
for our spatial discretization \citep[see e.g.,][]{schaal:2015, zhang:2010a, cockburn:1998, cockburn:1989}.
In this section we will briefly discuss our
implementation of the DG method, introducing notation and concepts.
For simplicity, we will focus on the one dimensional (1D) case. Recall that
we seek solutions to the Euler equations of hydrodynamics, constituting a
hyperbolic conservation law of the form

\beq
  \partial_{t} \mathbf{U} + \partial_{x} \mathbf{F}(\mathbf{U}) = 0
  \label{eq:ode}
\eeq
Where $\mathbf{U}$ is the evolved state vector and $\mathbf{F}(\mathbf{U})$ is
the flux. In order to solve Equation \eqref{eq:ode} numerically, we divide the computational
domain $D\subset \mathbb{R}$ into a disjoint union $\mathcal{T}$ of open elements
$\bK$ such that $D = \cup_{\bK \in \cT}\bK$. Each element $\bK$ is a box in the
coordinates
\beq
  \bK \in (\xL,x_{R}),
\eeq
where $\xL$ and $x_{R}$ are the left and right boundaries of the cell.
We let the approximation space $\mathbb{V}^{k}$ for the DG method
be polynomials of maximal degree $k$. Note that functions in $\mathbb{V}^{k}$
can be discontinuous across element interfaces. The DG problem is then to find
$\mathbf{U}_h \in \mathbb{V}^{k}$ which approximates $\mathbf{U}$ in Equation
\eqref{eq:ode}, such that $\forall$ $\phi \in \mathbb{V}^{k}$ and $\bK \in \cT$

\beq
  \partial_{t} \int_{\bK}\mathbf{U}_h\, \phi\, dx +
  \int_{\bK}\partial_{x}\, \mathbf{F(\mathbf{U})}\, \phi\, dx = 0.
\label{eq:semidiscreteDG_almost}
\eeq
Integrating the second term by parts, this becomes
\beq
   \partial_{t} \int_{\bK}\mathbf{U_{h}}\, \phi\, dx +
   \widehat{\mathbf{F}}(\mathbf{U}_h)\, \phi^{-} \big|_{x_{R}}
   + \widehat{\mathbf{F}}(\mathbf{U}_h)\, \phi^{+} \big|_{x_{L}} +
   \int_{\bK}\mathbf{F(\mathbf{U}_h)}\, \partial_{x}\phi\, dx = 0.
\label{eq:semidiscreteDG}
\eeq
In Eq.~\eqref{eq:semidiscreteDG}, $\widehat{\vect{F}}(\vect{U}_{h})$ is a
numerical flux approximating the flux on the boundary of $\bK$.
The numerical flux function is evaluated using values from
both boundaries of an element; i.e.,
\begin{equation}
  \widehat{\vect{F}}(\vect{U}_{h})=\vect{f}(\vect{U}_{h}(x^{-}),\vect{U}_{h}(x^{+})),
\end{equation}
where superscripts $-/+$
indicate that the function is evaluated to the immediate left/right of the
interface. We use the Harten-Lax-van Leer (HLL) flux \citep{harten:1983}
for all the numerical experiments presented in Section \ref{sec:results}.

In each element $\bK$, we use a nodal representation of the conserved variables
$\mathbf{U}$:

\begin{equation}
  \vect{U}(x,t)\approx
  \vect{U}_{h}(x,t)=\sum_{i=1}^{N=k+1}\vect{U}_{i}(t)\,\ell_{i}(x),
  \quad\text{where}\quad
  \ell_{i}(\eta)=
  \prod_{\substack{j=1\\j\ne i}}^{N=k+1}\f{\eta-\eta_{j}}{\eta_{i}-\eta_{j}}
  \label{eq:conservedNodalExpansion}
\end{equation}

are Lagrange polynomials defined on $I = \{ \eta : \eta \in (-0.5,0.5) \}$,
and are constructed to interpolate the node set
$S_{N}=\{\eta_{i}\}_{i=1}^{N}\subset I$. The spatial coordinate $x$ and the
reference coordinate $\eta$ are related by the
mapping $x(\eta)=\xL+(0.5+\eta)\,\Delta x$.
Then, for any $\eta_{j}\in S_{N}$, $\ell_{i}(\eta_{j})=\delta_{ij}$,
so that $\vect{U}_{h}(x(\eta_{j}),t)=\vect{U}_{j}(t)$.
We define the $M$-point quadrature $Q_{M}:C^{0}(I)\to\mathbb{R}$
with abscissas $\hat{S}_{M}=\{\eta_{q}\}_{q=1}^{M}$ and weights
$\{w_{q}\}_{q=1}^{M}$, normalized such that $\sum_{q=1}^{M}w_{q}=1$ in order to
evaluate the integrals in Eq.~\eqref{eq:semidiscreteDG}.
We use the $M$-point Legendre-Gauss quadrature, which is exact for
polynomials of degree $\le 2M-1$.
Then, if $P_{h}(x)$ is such a polynomial, we have
\beq
  \f{1}{\Delta x}\int_{\bK}P_{h}(x)\,dx=\int_{I}P_{h}(\eta)\,d\eta=\sum_{q=1}^{M}w_{q}\,P_{h}(\eta_{q}).
\eeq

We let $M=N$ and $S_{N}=\hat{S}_{N}$, which is a
spectral-type nodal collocation DG approximation \citep{bassi:2013}, which is exact
for Cartesian coordinates and Legendre-Gauss quadrature.
Inserting Eq.~\eqref{eq:conservedNodalExpansion} into Eq.~\eqref{eq:semidiscreteDG},
letting $\phi(x)=\ell_{k}(x)$, using the quadratures defined above,
we obtain
\begin{align}
  \pd{}{t}\int_{\bK}\vect{U}_{h}\,\phi\,dx
  &=w_{k}\,\pd{}{t}\vect{U}_{k}\,\Delta x
  \label{eq:timeDerivativeTerm}
\end{align}
for the time derivative piece, where $\Delta x = x_{R} - \xL$.
Similarly, the last term on the left-hand side of Eq.~\eqref{eq:semidiscreteDG}
becomes
\beq
  \int_{K}\vect{F}(\vect{U}_{h})\,\pderiv{\phi}{x}\,dx
  = \sum_{q=1}^{N}w_{q}\,\vect{F}(\vect{U}_{q})\,\pderiv{\ell_{k}}{\eta}(\eta_{q}).
  \label{eq:volumeTerm}
\eeq
Now we may combine Eqs.~\eqref{eq:timeDerivativeTerm}-\eqref{eq:volumeTerm} in
Eq.~\eqref{eq:semidiscreteDG} resulting in the semi-discrete form
\beq
  \deriv{\vect{U}_{k}}{t}
   =-\f{1}{w_{k}\Delta x}
  \Big\{
  \Big[\,
    \widehat{\vect{F}}\ell_{k}\big|_{x_{R}}
     -\widehat{\vect{F}}\ell_{k}\big|_{\xL}
  \,\Big]
   -\sum_{q=1}^{N}w_{q}\,\vect{F}(\vect{U}_{q})\,\pderiv{\ell_{k}}{\eta}(\eta_{q})
  \Big\}.
  \label{eq:semidiscreteDiscretized}
\eeq
We now have a system of ordinary
differential equations (ODEs), which may be evolved in time with an ODE solver.
In Section~\ref{sec:results} we use the third-order strong
stability-preserving Runge-Kutta (SSP-RK3) method \citep{shu:1988}.

\subsection{Slope Limiting}
\label{sec:Limiting}
A common feature of high-order numerical PDEs is unphysical oscillations in the
solutions around discontinuities.
It is therfore of great interest in the DG algorithm to implement slope limiting
of the polynomial $\mathbf{U}_h$. We use the total variation diminishing (TVD)
slope limiter \citep[see, e.g.,][]{cockburn:1998} in conjunction with the
troubled cell indicator discussed in \citet{fu:2017} to prevent excessive limiting
by flagging elements where limiting is needed. To do this, we need to reduce under- and
overshootings of the higher-order solution at cell boundaries compared to
the cell averages of neighbor cells. Recall from Eq.~\eqref{eq:conservedNodalExpansion}
that in each cell our solution is expressed in the nodal form. It is convenient,
however, for limiting purposes to express the solution in a modal representation:

\beq
\mathbf{U}_h (x,t) =
\sum_{l=1}^{N=k+1}c_{l}(t)\,P_{l}(x)
\eeq

\noindent where $P_{l}(x)$ are the Legendre polynomials. These
representations of the solution are equivalent by requiring weak equivalence

\beq
  \sum_{j=1}^{N}\int_{\bK} \left( \vect{U}_{j}(t)\, \ell_{j}(x) -
    c_{j}(t)\, P_{j}(x) \right) \phi\, dx = 0 \quad \forall\,\phi \in \mathbb{V}^{k}.
    \label{eq:NodalModal}
\eeq

\noindent Letting $\phi$ be the Lagrange interpolating polynomials $\ell_{i}$, we have

\beq
  \sum_{j=1}^{N}\int_{\bK}\ell_{i}(x)\, \ell_{j}(x)\, dx\, \vect{U}_{j}(t) =
  \sum_{j=1}^{N}\int_{\bK}\ell_{i}(x)\, P_{j}(x)\, dx\, c_{j}(t).
\eeq

\noindent Now letting $ M_{ij} = \int_{\bK}\ell_{i}(x)\, \ell_{j}(x)\, dx$ be the mass matrix,
$A_{ij} = \int_{\bK}\ell_{i}(x)\, P_{j}(x)\, dx$, $\bar{\vect{U}} = \{U_{1},...,U_{N}
\}$ be the nodal coefficients, and $\bar{\vect{c}} = \{c_{1},...,c_{N}\}$
be the modal coefficients, we see that the different representations are related by
a linear transformation

\beq
  M \bar{\vect{U}} = A \bar{\vect{c}}
\eeq
To perform slope limiting, we compare
the weight $c_{2}$, which is proportional to the first derivative of the
solution in the cell, to the neighboring cell averages by the following

\beq
  \mathcal{M}\,\widetilde{c}_{2} = \text{minmod}(\mathcal{M}c_{2},
    \beta_{\TVD}\mathcal{M}(c_{1}^{+} - c_{1}), \beta_{\TVD}\mathcal{M}(c_{1} - c_{1}^{-}))
  \label{eq:limiting}
\eeq

\noindent where $\widetilde{c}_{2}$ is the limited weight. The superscripts $-/+$
indicate the cell averages of the neighboring cells to the  immediate left/right
of the interface and $\mathcal{M}$ is a transformation matrix. For componentwise
limiting, we let $\mathcal{M}$ be the identity matrix. Notice that

\beq
%  \begin{align}
    \frac{1}{\Delta x}\int_{\bK}\vect{U}_{h}\,dx =
      \frac{1}{\Delta x} \sum_{l=1}^{N}\int_{\bK}P_{l}(x)\,\cdot 1\, dx\, c_{l}
      = \delta_{l1} c_{l} = c_{1}
  \label{eq:cellAverage}
%  \end{align}
\eeq
so $c_{1}$ is precisely the cell average. The minmod function in
Eq.~\eqref{eq:limiting} is defined as

\beq
  \text{minmod}(a_{1}, a_{2}, a_{3}) =
   \begin{cases}
      s\,\, \text{min}\{\abs{a_{1}}, \abs{a_{2}}, \abs{a_{3}}\} &
        s = \text{sign}(a_{1})= \text{sign}(a_{2})= \text{sign}(a_{3}) \\
      0 & \text{otherwise.}
   \end{cases}
\label{eq:minmod}
\eeq

The parameter $\beta_{\TVD}$ takes values in the closed interval $\left[1,2\right]$
and scales the strength of the limiting. A minimal $\beta_{\TVD}$ corresponds with
the total variation diminishing scheme but is more
dissipative than the maximal $\beta_{\TVD}$ case, which is potentially more oscillatory.
Increasing $\beta_{\TVD}$ puts more weight on the neighboring cell averages,
making the minmod function more likely to set $\widetilde{c}_{2} = c_{2}$, when
no limiting is applied. If instead $\widetilde{c}_{2} \neq c_{2}$, the solution
is truncated to first order

\beq
  \vect{U}_{h} \to \widetilde{\vect{U}}_{h} = c_{1}\, P_{1}(x) + \widetilde{c}_{2}\,P_{2}(x)
\eeq
and, by Eq.~\eqref{eq:minmod}, $\widetilde{c}_{2}$ can potentially be zero as well.
Notice that, by Eq.~\eqref{eq:cellAverage}, the cell average is not altered by
the limiting process, so the process is conservatory.

In order to determine where slope limiting is necessary, we use the
troubled cell indicator (TCI) \citep{fu:2017} to prevent excessive limiting

\begin{equation}
  I_{\bK}(G) = \f{\sum_{j}|G_{\bK}-G_{\bK}^{(j)}|}{\max_{j}|G_{\bK^{(j)}}^{(j)}|},
  \label{eq:tci}
\end{equation}

\noindent where $G\in\vect{G}\subseteq\vect{U}$ and $\vect{G} = \left(\rho,E\right)$.
Here, the sum in the numerator and the
max in the denominator are taken over the neighboring elements sharing a
boundary with target element $\bK$, $G_{\bK}$ is the cell average in $\bK$,
$G_{\bK}^{(j)}$ is the cell average
computed by extrapolating the polynomial representation from the neighboring
element $\bK^{(j)}$ into $\bK$, and $G_{\bK^{(j)}}^{(j)}$
is the cell average native to neighbor element $\bK^{(j)}$. An element is flagged
for limiting if, for any $G\in\vect{G}$, $I_{\bK}(G)>C_{\TCI}$, where
$C_{\TCI}$ is a defined threshold.

\subsection{Time Integration}
\label{sec:TimeInt}
Eq.~\eqref{eq:semidiscreteDiscretized} presented a system of ODEs that must
be evolved in time. We use the third-order strong stability-preserving
Runge-Kutta (SSP-RK) scheme \citep{shu:1988}.
The general $s$-stage Runge-Kutta time stepping algorithm, including
the limiting process, can be summarized as in \citet{cockburn:2001}:
\begin{itemize}
  \item[1.] Set $\bar{\vect{U}}^{(0)} = \bar{\vect{U}}^{n}$,
  \item[2.] For $i=1,\ldots,s$ compute:
  \begin{equation}
  \small  \bar{\vect{U}}^{(i)}
    = \Lambda^{\TVD}\Big\{\sum_{j=0}^{i-1}\alpha_{ij}\,\bar{\vect{U}}^{(j)}+\beta_{ij}\,\Delta t\,\bar{\vect{F}}\big(\bar{\vect{U}}^{(j)}\big)\Big\},
    \label{eq:rkStages}
  \end{equation}
  \item[3.] Set $\bar{\vect{U}}^{n+1}=\bar{\vect{U}}^{(s)}$.
\end{itemize}
Above, the TVD limiter limiter preventing unphysical
states is denoted by the operator $\Lambda^{\TVD}\{\}$. The SSP-RK3 coefficients $\alpha_{ij}$
and $\beta_{ij}$ may be found in Table~2.1 in \citet{cockburn:2001a}. For each step
in Eq.~\eqref{eq:rkStages}, the TVD limiter is applied to elements flagged
by the troubled cell indicator (where $I_{\bK} > C_{\TCI}$).
This algorithm is subject to a timestep stability condition $\Delta t_{\TVD}$
where $|\lambda|$ is the magnitude of the largest eigenvalue of the flux Jacobian.
Then (e.g., \cite{cockburn:2001a}):
\begin{equation}
  \Delta t_{\TVD} \le \f{1}{d}\,\f{(\Delta x/|\lambda|)}{(2k+1)}
  \label{eq:timestepRestrictions}
\end{equation}
where d is the dimension. For $d=1,k=2$ (third-order scheme) the effective Courant-Friedrichs-Lewy (CFL) factor is $0.2$.
In the numerical experiments presented in Section~\ref{sec:results} we use
a more conservative timestep restriction of $\Delta t = 0.1 (\Delta x/|\lambda|)$.


\subsection{Characteristic Decomposition}
\label{sec:characteristicDecomp}
Experience has shown that the slope limiting described in the previous section
is more effecient when performed on the so-called `characteristic variables`
as opposed to the conserved variables $\mathbf{U}_h$
\citep[see, e.g.,][for a description]{cockburn:1998}.
Inspired by theory for linear systems, we begin by
rewriting Eq.~\eqref{eq:conservation} in the quasi-linear form as follows
\beq
  \pderiv{\mathbf{U}}{t}
  + \pderiv{\mathbf{F(\mathbf{U})}}{\mathbf{U}}\pderiv{\mathbf{U}}{x}
  = 0.
  \label{eq:charEq}
\eeq
Because the Euler equations form a system of hyperblic
partial differential equations \citep[see, e.g.,][]{leveque:1992}, we can decompose the
Jacobian of the flux vector as

\beq
  \pderiv{\mathbf{F(\mathbf{U})}}{\mathbf{U}} =
  \mathcal{R} \Lambda \mathcal{R}^{-1},
\eeq
where the columns of $\mathcal{R}$ contain the right eigenvectors of the Jacobian,
the rows of $\mathcal{R}^{-1}$ contain the left eigenvectors, and
$\Lambda$ is a diagonal matrix containing the eigenvalues of the Jacobian.
For hyperbolic systems, the eigenvalues are real and the eigenvectors form a
complete set \citep[see e.g.,][]{leveque:1992}.
At this point, we will introduce the characteristic variable
$\mathbf{w} = \mathcal{R}^{-1}\mathbf{U}$. Multiplying both sides of Equation
\eqref{eq:charEq} by $\mathcal{R}^{-1}$ and simplifying, we are able to linearize the system of equations to
a system of advection equations

\beq
  \pderiv{\mathbf{w}}{t} +
  \Lambda \pderiv{\mathbf{w}}{x}
  = 0. \\
\eeq
Solutions to these now decoupled advection equations are far simpler to obtain than the
previous equations. Recall in Eq.~\eqref{eq:minmod} that the limiting cell
averages were transformed by a matrix $\mathcal{M}$. If we let
$\mathcal{M}=\mathcal{R}^{-1}$ then limiting is performed on the characteristic variables.
Limiting may then be applied to dampen oscillations in the
solutions for the characteristic variables $\mathbf{w}$, which are then
transformed back to the solution of the conserved variables $\mathbf{U}$
\citep[see e.g.,][for a description]{cockburn:1998, schaal:2015a}.
While this
process of characteristic limiting has been done for the ideal EOS
\citep[][]{cockburn:1998}, we want to extend this process to the case of a
tabulated nuclear matter EOS.


\section{Jacobian and Eigensystem for the Nuclear Case}
\label{sec:eigen}

We assume that pressure $P = P(\tau, \epsilon, D_{e})$, where $\tau = \frac{1}{\rho}$.
Let the vector of conserved variables be $\textbf{U} = \{\rho, m_1, m_2, m_3, E, D_{e}\}$,
where $m_i = \rho v_i$ and $D_{e} = \rho y_{e}$.
The flux vector is $\textbf{F}(\textbf{U}) =
\{m_{1}, m_{1}^{2}\tau + P, m_{1}m_{2}\tau, m_{1}m_{3}\tau,
(E+P)m_{1}\tau, D_{e}m_{1}\tau\}$. The Jacobian of the flux vector is given by


%\begin{table*}[ht!]

\begin{align}
  \centering
	\pderiv{\mathbf{F}(\mathbf{U})}{\mathbf{U}}
	= \left[
		\begin{array}{cccccc}
			0 & 1 & 0 & 0 & 0 & 0 \\
			-v_{1}^{2} -P_{\tau}\tau^{2} - P_{\epsilon}\tau(\epsilon - \frac{v^{2}}{2}) & v_{1}(2-P_{\epsilon}\tau)  & -P_{\epsilon}v_{2}\tau & -P_{\epsilon}v_{3}\tau  & P_{\epsilon}\tau  & P_{D_{e}} \\
			-v_{1}v_{2} & v_2 & v_1 & 0 & 0 & 0 \\
			-v_{1}v_{3} & v_{3} & 0 & v_{1} & 0 & 0 \\
			v_{1}(-H - P_{\tau}\tau^{2} -P_{\epsilon}\tau(\epsilon - \frac{v^{2}}{2})) & H - P_{\epsilon}v_{1}^{2}\tau  & -P_{\epsilon}v_{1}v_{2}\tau & -P_{\epsilon}v_{1}v_{3}\tau  & v_{1}(1+P_{\epsilon}\tau) & v_{1} P_{D_{e}} \\
			-v_{1} y_{e} & y_{e} & 0 & 0 & 0 & v_{1} \\
		\end{array}
    \right]
\end{align}
%\caption{}
%\end{table*}


\noindent Where $H=(E+P)\tau$ is the specific enthalpy and
\beq
%  \begin{align}
    P_{\epsilon}  = \left(\frac{\partial{P}}{\partial{\epsilon}}\right)_{\tau,D_{e}}, \quad
    P_{D_e}  = \left(\frac{\partial{P}}{\partial{D_{e}}}\right)_{\tau, \epsilon}, \quad
    P_{\tau}  = \left(\frac{\partial{P}}{\partial{\tau}}\right)_{\epsilon, D_{e}}
%  \end{align}
\eeq
are the necessary thermodynamic derivatives. Expressions for these derivatives
in terms of the table variables $\rho, T, \text{and}~y_{e}$ are given in Appendix~\ref{appendix:deriv}.
The eigenvalues of the flux Jacobian are given by the diagonal matrix

\begin{align}
\Lambda =
\begin{bmatrix}
  v_{1} - c_{s} & 0 & 0& 0& 0& 0 \\
  0 & v_{1} & 0 & 0 & 0 & 0      \\
  0 & 0 & v_{1} & 0 & 0 & 0      \\
  0 & 0 & 0 & v_{1} & 0 & 0      \\
  0 & 0 & 0 & 0 & v_{1} & 0      \\
  0 & 0 & 0 & 0 & 0 & v_{1} + c_{s}
\end{bmatrix}
\end{align}
where $c_{s} = \sqrt{\Gamma P \tau}$, with
$\Gamma = \left(\tau (P P_{\epsilon} - P_{\tau}) + P_{D_e} y_{e} \tau^{-1}\right) P^{-1}$, is
the local sound speed. In the case of the general EOS $P=P(\tau,\epsilon)$,
this reduces to the sound speeds given in \citet{colella:1985}.
The right eigenvectors are then given by the column vectors of the following matrix
\begin{align*}
  \mathcal{R}_{1} =
  \left[
  \begin{array}{cccccc}
   1 & 0 & 1 & 1 & 0 & 1 \\
   v_{1}-c & 0 & v_{1} & v_{1} & 0 & c+v_{1} \\
   v_{2} & 1 & 0 & 0 & 0 & v_{2} \\
   v_{3} & 0 & 0 & 0 & 1 & v_{3} \\
   h-c v_{1} & v_{2} & \beta & 0 & v_{3} & h+c v_{1} \\
   y_{e}  & 0 & 0 & \frac{\tau  \chi }{2 P_{D_{e}}} & 0 & y_{e}  \\
  \end{array}
  \right]
\end{align*}
where the following definitions have been used:
$h_{n} = \frac{c^2}{P_{\epsilon}\tau} + k $, $k = \frac{-y_{e} P_{D_{e}} \tau^{-1}
+ P_{\epsilon} (\frac{1}{2}v^2 + \epsilon) + P_{\tau}\tau}{P_{\epsilon}}$,
$\delta_{1} = v_{1}^{2}-v_{2}^{2}-v_{3}^{2}$,
$\chi = P_{\epsilon} ( \delta_{1} + 2\epsilon) + 2P_{\tau}\tau$, and
$\beta = \frac{1}{2} (\delta_{1}+2 \epsilon +\frac{2 P_{\tau} \tau }{P_{\epsilon}})$.
The left eigenvectors are given by the row vectors of the inverse matrix $\mathcal{L}_{1} = \mathcal{R}_{1}^{-1}$

\begin{align*}
  \mathcal{R}_{1}^{-1} = \frac{1}{c^2}
  \left[
  \begin{array}{cccccc}
   \frac{1}{4} (2 c  v_{1}+\omega ) & \frac{1}{2} (-c- \phi_{1} ) & -\frac{1}{2} \phi_{2}
    & -\frac{1}{2} \phi_{3}  & \frac{P_{\epsilon} \tau }{2} & \frac{P_{D_{e}}}{2}
     \\
   -\frac{v_{2} \omega }{2} & \phi_{1} v_{2}  & c^2+\phi_{2} v_{2}  &
     \phi_{3} v_{2}  & -\phi_{2}  & -P_{D_{e}} v_{2}
     \\
   \frac{2 \chi  c^2+\alpha  \omega \tau^{-1} }{2 \chi } & -\frac{\phi_{1} \alpha  }{\chi \tau } &
     -\frac{\phi_{2} \alpha  }{\chi \tau } & -\frac{\phi_{3} \alpha }{\chi \tau } &
     \frac{P_{\epsilon} \alpha }{\chi } & \frac{P_{D_{e}} \left(\alpha -2 c^2\right)}{\tau \chi }
      \\
   -\frac{y_{e} P_{D_{e}} \omega }{\chi \tau } & \frac{2 y_{e} P_{D_{e}} \phi_{1} }{\chi \tau } & \frac{2 y_{e} P_{D_{e}}
     \phi_{2} }{\chi \tau} & \frac{2 y_{e} P_{D_{e}} \phi_{3} }{\chi \tau} & -\frac{2 y_{e}
     P_{D_{e}} P_{\epsilon} }{\chi } & \frac{2 P_{D_{e}} \left(c^2-y_{e} P_{D_{e}} \right)}{\tau  \chi }
      \\
   -\frac{v_{3} \omega }{2} & \phi_{1} v_{3}  & \phi_{2} v_{3}   &
     c^2+\phi_{3} v_{3}  & -\phi_{3}  & -P_{D_{e}} v_{3}
      \\
   \frac{1}{4} (\omega -2 c  v_{1}) & \frac{1}{2} (c-\phi_{1} ) & -\frac{1}{2} \phi_{2}
       & -\frac{1}{2} \phi_{3}  & \frac{P_{\epsilon} \tau }{2} & \frac{P_{D_{e}}}{2}
     \\
  \end{array}
  \right]
\end{align*}
where $\phi_{i} = P_{\epsilon}\,\tau\, v_{i}$,
$\omega = \tau\, (P_{\epsilon}\,(v^2 - 2\epsilon) - 2\,P_{\tau}\,\tau)$, and
$\alpha = 2 y_{e} P_{D_{e}} - \tau \chi$.
The left and right eigenvectors presented here are similar in form to those
presented in \citet{schaal:2015a}, but are considerably more complicated due to the
restriction of a nuclear EOS. Having an explicit form for these matrices is
desirable as it eliminates the need to numerically diagonalize the flux
Jacobian\footnote{These matrices and their derivations are available in Mathematica
notebooks at \url{https://github.com/AstroBarker/dgHydro-nuc_charDecomp}}.

\section{Preliminary Numerical Results}
\label{sec:results}
In this section we present preliminary numerical results obtained with the DG
method implemented in \thornado\, leveraging use of the characteristic
decomposition. Unless otherwise noted, we use the SFHo EOS of
nuclear matter \citep{steiner:2013} now commonly used in high-fidelity CCSN simulations
due to its consistency with neutron star mass and radius observations.
These tests serve to gauge the performance improvement of the
DG characteristic decomposition implementation on a set of benchmarks as an initial
assessment of its suitability for future CCSN simulations.

\subsection{Riemann Problem}
\label{sec:results:1}
We present a 1D Riemann problem inspired by the classic 1D Sod shock tube Riemann problem \citep{sod:1978} in
Cartesian coordinates. This is an ideal first test as it is a fairly simple
scenario but contains both a contact discontinuity and a shock to test the
efficiency of our limiting scheme, though due to the nature of our EOS, an analytic
solution does not exist. The computational domain is D = [-5,5]~km
with a discontinuity initially located at $x = 0$~km separating the left and right states

\begin{align*}
  \mathbf{U}_{L} &= (10^{12}~\text{g~cm}^{-3}, 0\,, 2.703\text{x}10^{32}~\text{ergs~cm}^{-3}, 0.4\text{x}10^{12})^T\,\,\, \\
  \mathbf{U}_{R} &= (1.25\text{x}10^{11}~\text{g~cm}^{-3}, 0\, , 2.822\text{x}10^{31}~\text{ergs~cm}^{-3}, 0.375\text{x}10^{11})^T.
\end{align*}

\noindent The test is run until $t = 0.025$~ms with 100 elements
using $C_{\TCI} = 0.2$ and $\beta_{\TVD} = 2.0$ with a third-order time
integration SSP-RK3 scheme. Results for density, pressure, velocity, and electron fraction
are plotted in the upper left, upper right, lower left, and lower right panels of
Figure~\ref{fig:SodSedovOptimal}, respectively, compared to a reference run using
10000 elements, third-order time integration, and first order spacial discretization.
Using the characteristic limiting described
above, the DG method captures the features of the solution,
resolving well the contact discontinuity, shock, and rarefraction wave
located at about $x=2$~km, $x=4$~km, and from $x=-3$~km to $x=0$~km, respectively, without introducing
noticeable oscillations in the numerical solutions near the discontinuities.
In Figure~\ref{fig:shock}
we plot elements in the $xt$-plane flagged by the troubled cell indicator for
limiting, showing that the troubled cell indocator flagged cells around both
the shock and contact discontinuity for limiting and maintained that
for the full time of the test.

\begin{figure}[h!]
  \centering
  \includegraphics[width=32pc,height=30pc]{optimal.png}
  \centering
  \caption{\label{fig:SodSedovOptimal} Numerical solution of the Riemann problem using
    100 elements and $C_{\TCI} = 0.2$ and $\beta_{\TVD} = 2.0$ with a third-order
    time integration SSP-RK3 scheme for density (upper left), pressure (upper right),
    velocity (lower left), and electron fraction (lower right)
    compared with a reference solution (black) using 10000 elements.}
\end{figure}
%\clearpage
\begin{wrapfigure}{L}{2.8in}
  \includegraphics[width=18pc]{optimal_shock.png}
  \caption{$xt$-plane of elements flagged for limiting by the troubled cell indicator
  in the Riemann problem of Section~\ref{sec:results:1}.}
  \label{fig:shock}
\end{wrapfigure}

\subsection{Optimal Limiting Parameters}
\label{sec:param}
In order to determine the optimal limiting parameters $C_{\TCI}$ and $\beta_{\TVD}$,
we performed a suite of sixteen simulations of Sod's problem with various
limiting parameters with $\beta_{\TVD}\in\left[1.0,2.0\right]$ and
$C_{\TCI}\in\left[0.0,0.2\right]$. For each $\left(\beta_{\TVD},C_{\TCI}\right)$ pair,
we computed the relative error and total variation \
\beq
  \varepsilon(\mathbf{U}) = \frac{1}{N}\sum_{\bK}\left\lvert\frac{\vect{U}_{h}^{\text{ref}}(x_{q}) - \vect{U}_{h}(x_{q})}{\vect{U}_{h}^{\text{ref}}(x_{q})}\right\rvert \quad\quad
  TV = \sum_{k}\abs{ \bar{\vect{U}}_{k+1} - \bar{\vect{U}}_{k} }
\eeq
 in density and electron
fraction at $t = 0.025$ms, where $\vect{U}_{h}^{\text{ref}} $ is the reference
solution, $\bar{\vect{U}}_{k+1}$ is the cell average is cell $k$, and $x_{q}$ are
numerical quadrature points.
Results are plotted in Figure~\ref{fig:landscape}.
We find that, as expected, increasing $\beta_{\TVD}$ tends to monotonically
decrease the relative error while (non-monotonically) increasing the total variation due to the
limiter allowing for more oscillations. Increasing $C_{\TCI}$ tends to decrease
the relative error and increase the total variation.
We have selected
$\beta_{\TVD}=1.75$ and $C_{\TCI}=0.1$ to be the optimal parameters providing
the best combination of relative error and total variation reduction. Unless
otherwise noted, all following results will use this combination of limiting
parameters.

\begin{figure}[b]
  \centering
  \includegraphics[width=32pc, height=27.5pc]{./figures/error_tv_landscape.png}
  \caption{\label{fig:landscape} Error and total variation landscapes for various
  values of the limiting parameters $C_{TCI}$ and $\beta_{TVD}$. The top row shows
  total variation in density (left) and electron fraction (right). The bottom
  row show relative error in density (left) and electron fraction (right).}
\end{figure}

\subsection{Improvement From Componentwise Limiting}
\label{sec:optimal}
The motivation for limiting on the characteristic variables is the potential
improvement from componentwise limiting. In Figure~\ref{fig:SodSedovCW} we plot the
solution at $t = 0.025$~ms using 100 elements employing both characteristic
limiting (blue) and componentwise limiting (red) compared to the reference
solution computed with 10000 elements. We observe, for the componentwise limiting,
noticable oscillations in density around the contact discontinuity as well
as less resolved discontinuities. The electron fraction displays very
large oscillations at the left side of the contact discontinuity, up to
$\pm 0.02$ in the cell averaged electron fraction. The solutions with
characteristic limiting are very close to the reference solution, presenting
a significant improvement over componentwise limiting.
This makes characteristic
limiting particularly appealing for CCSN simulations where the electron fraction
plays a critical role in the explosion dynamics.

% \begin{figure}[h]
%   \centering
%   \includegraphics[width=32pc, height=32pc]{./figures/error_tv_landscape.png}
%   \caption{\label{fig:landscape} Error - total variation landscapes for various
%   values of the limiting parameters $C_{TCI}$ and $\beta_{TVD}$.}
% \end{figure}

\begin{figure}[h!]
  \centering
  \includegraphics[width=36pc]{./figures/characteristic_cw.png}
  \caption{\label{fig:SodSedovCW} Numerical solution of Sod's problem using
  100 elements with characteristic limiting (blue) and componentwise limiting
  (red) for density (upper left), pressure (upper right),
  velocity (lower left), and electron fraction (lower right)
  compared to a reference solution (black) using 10000 elements.compared with a reference solution using 10000 elements.}
\end{figure}

\vspace{1cm}
\subsection{High Density Regime}
This test is included to verify the performance improvements of characteristic
limiting in a higher density regime. It is designed with the Sod problem in mind.
The computational domain is $D = [-5,5]$~km
with a discontinuity initially located at $x = 0$~km separating the left and right states

\begin{align*}
  \mathbf{U}_{L} &= (10^{13}~\text{g~cm}^{-3}, 0\,, 3.712\text{x}10^{32}~\text{ergs~cm}^{-3}, 0.15\text{x}10^{12})^T\,\,\, \\
  \mathbf{U}_{R} &= (1.25\text{x}10^{12}~\text{g~cm}^{-3}, 0\, , 3.015\text{x}10^{31}~\text{ergs~cm}^{-3}, 0.169\text{x}10^{12})^T.
\end{align*}

\noindent The test is run until $t = 0.05$~ms with 100 elements
using $C_{\TCI} = 0.1$ and $\beta_{\TVD} = 1.2$. We chose a lower $\beta_{\TVD}$
here, as larger values resulted in unphysical internal energies outside of the
EOS table causing the test to fail, motivating the development of a
positivity limiter compatible with a tabulated nuclear matter EOS \citep{zhang:2010a}.
Inital state values were chosen to be consistent with
high-fidelity CCSN simulation code \chimera\, \citep{bruenn:2018} data by first
choosing the left and right densities to be an order of magnitude larger than in
Section~\ref{sec:results:1}, and then finding temperatures and electron fractions
consistent with those densities from \chimera\, data, and computing the other
state variables from the EOS table.
Results are plotted in
Figure~\ref{fig:SodSedovHighDens}. Our method captures the main features
of the solution, but introduces more oscillations than in the lower density
case, even when using the characteristic limiter.
We notice an undershooting of the right side
of the contact discontinuity in the electron fraction not previously present
warranting further investigation.

\begin{figure}[h!]
  \centering
  \includegraphics[width=36pc]{./figures/highDens.png}
  \caption{\label{fig:SodSedovHighDens} Results for the high density Sod-like problem
  computed using 100 elements and $C_{\TCI} = 0.1$ and $\beta_{\TVD} = 1.2$ at $t=0.05$ms.}
\end{figure}

\subsection{EOS Resolution Dependence}
\label{sec:eosRes}
In order to test the sensitivity of the limiter on the EOS table resolution,
we repeated the tests in Section~\ref{sec:optimal} with a higher resolution
SFHo EOS table. As before, tests are computed using 100 elements until
$t = 0.025$ms with characteristic limiting and the optimal limiting parameters determined in
Section~\ref{sec:param} and compared to a reference solution computed using
10000 elements. Results for both tables are plotted in Figure~\ref{fig:SodSedovSFHoRes}.
We find no sensitivity to the table resolution, with the exception of
minor differences in density to the left of the contact discontinuity.
\begin{figure}[h!]
  \centering
  \includegraphics[width=36pc]{./figures/eos_res.png}
  \caption{\label{fig:SodSedovSFHoRes} Comparison of numerical solutions to
  Sod's problem using 100 elements with characteristic limiting and two
  different EOS table resolutions (low: red, high: black) for density (upper left), pressure (upper right),
  velocity (lower left), and electron fraction (lower right)
  compared with a reference solution (black) using 10000 elements and the low resolution EOS table.}
\end{figure}


\subsection{EOS Table Sensitivity}
In order to probe the sensitivity of the method to the EOS table, we repeated
the tests conducted in Section~\ref{sec:optimal} using three different EOS tables:
the high resolution SFHo table from Section~\ref{sec:eosRes}, the SFHx table, and the DD2 table.
As before, tests are computed using 100 elements until
$t = 0.025$ms with characteristic limiting and the optimal limiting parameters determined in
Section~\ref{sec:param}. We find nearly no sensitivity to the EOS table used in
the low density regime, with the exception of very small variations in density
to the left of the contact discontinuity. Future work includes testing the
sensitivity to the EOS table in higher density regimes.
\begin{figure}[h!]
  \centering
  \includegraphics[width=36pc]{./figures/eos_all.png}
  \caption{\label{fig:SodSedovEOS} Comparison of numerical solutions to
  Sod's problem using 100 elements with characteristic limiting and three
  different EOS tables for density (upper left), pressure (upper right),
  velocity (lower left), and electron fraction (lower right).
  }
\end{figure}
%\FloatBarrier

\section{Conclusions}

We have presented preliminary developments and numerical results for a
characteristic limiter to be used with solvers of the non-relativistic
Euler equations of gas dynamics in the {\bf t}oolkit for {\bf h}igh-{\bf or}der
{\bf n}eutrino-r{\bf ad}iation hydr{\bf o}dynamics (\thornado). We presented analytic
forms for the diagonalizating matrices for the flux Jacobian of the Euler equations
consistent with a tabulated nuclear EOS.
The results presented from a suite of 1D test problems
demonstrate the superior performance of the characteristic limiter
compared to the componentwise limiter. Moreover, after performing a parameter
study on the limiter parameters, we found optimal limiting
parameters for use with the Sod problem. While the optimal limiting parameters
tend to be problem dependent, we hope to investigate if such an optimal
parameter exists for CCSN applications. Tests with various EOS tables and
table resolutions showed little to no sensitivity to the table or its
resolution in the tests studied. For future work, we will extend these studies
to other higher density regimes applicable to the CCSN environment.
Planned near-future work on the nuclear equation of state compatibility of
\thornado\, includes the development of a positivity limiter and extension of
the characteristic limtier to handle multi-dimensional, relativistic, and curvilinear problems.
Because of the superior performance of characteristic limiting, we hope that
this work will improve the fidelity of CCSN simulations.

\acknowledgments{
Eirik Endeve and Anthony Mezzacappa acknowledge support from the NSF Gravitational Physics Program (NSF-GP 1505933 and 1806692).
}

\software{
  \href{https://matplotlib.org/}{Matplotlib} \citep{hunter:2007a},
  \href{http://www.numpy.org/}{NumPy} \citep{walt:2011},
  \href{https://www.scipy.org/}{SciPy} \citep{jones:2001},
  }

\appendix

\section{Derivatives}
\label{appendix:deriv}
Derivatives of pressure with respect to $\tau$, $\epsilon$, and n. \\
\begin{footnotesize}
\begin{eqnarray}
	\frac{\partial{P}}{\partial{\epsilon}} &=& \left(\frac{\partial{\epsilon}}{\partial{T}}\right)^{-1}\left(\frac{\partial{P}}{\partial{T}}\right) \\
	\frac{\partial{P}}{\partial{n}} &=& \frac{m_B}{\rho} \left[ \frac{\partial{P}}{\partial{y_e}} -
          \frac{\partial{\epsilon}}{\partial{y_e}}\left(\frac{\partial{\epsilon}}{\partial{T}}\right)^{-1}\left(\frac{\partial{P}}{\partial{T}}\right)\right]\\
	\frac{\partial{P}}{\partial{\tau}} &=& -\tau^{-2} \left[ \frac{\partial{P}}{\partial{\rho}} - \frac{y_e}{\rho} \left( \frac{\partial{P}}{\partial{y_e}} -
          \frac{\partial{\epsilon}}{\partial{y_e}} \left(\frac{\partial{\epsilon}}{\partial{T}}\right)^{-1}\left(\frac{\partial{P}}{\partial{T}}\right)\right) -
          \frac{\partial{\epsilon}}{\partial{\rho}}\left(\frac{\partial{\epsilon}}{\partial{T}}\right)^{-1}\left(\frac{\partial{P}}{\partial{T}}\right)\right]
\end{eqnarray}
\end{footnotesize}


\section{3D}
\label{appendix:3d}
3d 3d 3d
put Astronum Eq 15 here as well, restate U = (...)
For the non-relativistic Euler equations, the state and flux vectors are given by
\begin{equation}
  \vect{U}=\big(\rho,\rho v_{j},E\big)^{T},~
  \vect{F}^{i}=\big(\rho v^{i},\Pi^{i}_{~j},(E+p)v^{i}\big)^{T},~
\end{equation}
where $\rho$ and $v^{i}$ are the mass density and components of the fluid three-velocity, respectively.
The stress tensor is $\Pi^{i}_{~j}=\rho u^{i} u_{j}+p\delta^{i}_{~j}$, and $E=e+\f{1}{2}\rho u_{i} u^{i}$ is the fluid (internal plus kinetic) energy density.

%\begin{footnotesize}
	\begin{align}
	  \centering
		\pderiv{\mathbf{F}^{1}(\mathbf{U})}{\mathbf{U}}
		= \left[
			\begin{array}{cccccc}
				0 & 1 & 0 & 0 & 0 & 0 \\
				-v_{1}^{2} -P_{\tau}\tau^{2} - P_{\epsilon}\tau(\epsilon - \frac{v^{2}}{2}) & v_{1}(2-P_{\epsilon}\tau)  & -P_{\epsilon}v_{2}\tau & -P_{\epsilon}v_{3}\tau  & P_{\epsilon}\tau  & P_{D_{e}} \\
				-v_{1}v_{2} & v_2 & v_1 & 0 & 0 & 0 \\
				-v_{1}v_{3} & v_{3} & 0 & v_{1} & 0 & 0 \\
				v_{1}(-H - P_{\tau}\tau^{2} -P_{\epsilon}\tau(\epsilon - \frac{v^{2}}{2})) & H - P_{\epsilon}v_{1}^{2}\tau  & -P_{\epsilon}v_{1}v_{2}\tau & -P_{\epsilon}v_{1}v_{3}\tau  & v_{1}(1+P_{\epsilon}\tau) & v_{1} P_{D_{e}} \\
				-v_{1} y_{e} & y_{e} & 0 & 0 & 0 & v_{1} \\
			\end{array}
	    \right]
	\end{align}

	\begin{align}
		\centering
		\pderiv{\mathbf{F}^{2}(\mathbf{U})}{\mathbf{U}}
		= \left[
			\begin{array}{cccccc}
				0 & 0 & 1 & 0 & 0 & 0 \\
        -v_{1}v_{2} & v_{2} & v_{1} & 0 & 0 & 0 \\
				-v_{2}^{2} -P_{\tau}\tau^{2} - P_{\epsilon}\tau(\epsilon - \frac{v^{2}}{2}) & -P_{\epsilon}v_{1}\tau &
				  v_{2}(2-P_{\epsilon}\tau) &  -P_{\epsilon}v_{3}\tau & P_{\epsilon}\tau & P_{D_{e}} \\
				-v_{2}v_{3} & 0 & v_{3} & v_{2} & 0 & 0 \\
				v_{2}(-H - P_{\tau}\tau^{2} -P_{\epsilon}\tau(\epsilon - \frac{v^{2}}{2})) &
				  - P_{\epsilon}v_{1}v_{2}\tau & H - P_{\epsilon}v_{2}^{2}\tau &
				  - P_{\epsilon}v_{2}v_{3}\tau & v_{2}(1+P_{\epsilon}\tau) & v_{2}P_{D_{e}} \\
				-v_{2}y_{e} & 0 & y_{e} & 0 & 0 & v_{2}\\
			\end{array}
			\right]
	\end{align}

	\begin{align}
		\centering
		\pderiv{\mathbf{F}^{3}(\mathbf{U})}{\mathbf{U}}
		= \left[
			\begin{array}{cccccc}
				0 & 0 & 0 & 1 & 0 & 0 \\
        -v_{1}v_{3} & v_{3} & 0 & v_{1} & 0 & 0 \\
				-v_{2}v_{3} & 0 & v_{3} & v_{2} & 0 & 0 \\
				-v_{3}^{2} -P_{\tau}\tau^{2} - P_{\epsilon}\tau(\epsilon - \frac{v^{2}}{2}) & -P_{\epsilon}v_{1}\tau &
				  -P_{\epsilon}v_{2}\tau &  v_{3}(2 - P_{\epsilon}\tau) & P_{\epsilon}\tau & P_{D_{e}} \\
				v_{3}(-H - P_{\tau}\tau^{2} -P_{\epsilon}\tau(\epsilon - \frac{v^{2}}{2})) &
				 - P_{\epsilon}v_{1}v_{3}\tau & -P_{\epsilon}v_{2}v_{3}\tau & H - P_{\epsilon}v_{3}^{2}\tau &
				  v_{3}(1+P_{\epsilon}\tau) & v_{3}P_{D_{e}} \\
				-v_{3}y_{e} & 0 & y_{e} & 0 & 0 & v_{3}\\
			\end{array}
			\right]
	\end{align}

The eigenvalues of the flux Jacobian are given by the diagonal matrix

\begin{align}
\Lambda_{i} =
\begin{bmatrix}
  v_{i} - c_{s} & 0 & 0& 0& 0& 0 \\
  0 & v_{i} & 0 & 0 & 0 & 0      \\
  0 & 0 & v_{i} & 0 & 0 & 0      \\
  0 & 0 & 0 & v_{i} & 0 & 0      \\
  0 & 0 & 0 & 0 & v_{i} & 0      \\
  0 & 0 & 0 & 0 & 0 & v_{i} + c_{s}
\end{bmatrix}
\end{align}
where $c_{s} = \sqrt{\Gamma P \tau}$, with
$\Gamma = \left(\tau (P P_{\epsilon} - P_{\tau}) + P_{D_e} y_{e} \tau^{-1}\right) P^{-1}$, is
the local sound speed.

The right eigenvectors are then given by the column vectors of the following matrices
\begin{align*}
  \mathcal{R}_{1} =
  \left[
  \begin{array}{cccccc}
    1 & 0 & 1 & 1 & 0 & 1 \\
    v_{1}-c & 0 & v_{1} & v_{1} & 0 & c+v_{1} \\
    v_{2} & 1 & 0 & 0 & 0 & v_{2} \\
    v_{3} & 0 & 0 & 0 & 1 & v_{3} \\
    h-c v_{1} & v_{2} & \beta_{1} & 0 & v_{3} & h+c v_{1} \\
    y_{e}  & 0 & 0 & \frac{\tau  \chi_{1} }{2 P_{D_{e}}} & 0 & y_{e}  \\
  \end{array}
  \right]
\end{align*}

\begin{align*}
  \mathcal{R}_{2} =
  \left[
  \begin{array}{cccccc}
    1 & 0 & 1 & 1 & 0 & 1 \\
    v_{1} & 1 & 0 & 0 & 0 & v_{1} \\
    v_{2}-c & 0 & v_{2} & v_{2} & 0 & c+v_{2} \\
    v_{3} & 0 & 0 & 0 & 1 & v_{3} \\
    h-c v_{2} & v_{1} & \beta_{2} & 0 & v_{3} & h+c v_{2} \\
    y_{e}  & 0 & 0 & \frac{\tau  \chi_{2} }{2 P_{D_{e}}} & 0 & y_{e}  \\
  \end{array}
  \right]
\end{align*}


where the following definitions have been used:
$h_{n} = \frac{c^2}{P_{\epsilon}\tau} + k $, $k = \frac{-y_{e} P_{D_{e}} \tau^{-1}
+ P_{\epsilon} (\frac{1}{2}v^2 + \epsilon) + P_{\tau}\tau}{P_{\epsilon}}$,
$\delta_{1} = v_{1}^{2}-v_{2}^{2}-v_{3}^{2}$,
$\chi_{1} = P_{\epsilon} ( \delta_{1} + 2\epsilon) + 2P_{\tau}\tau$, and
$\beta_{1} = \frac{1}{2} (\delta_{1}+2 \epsilon +\frac{2 P_{\tau} \tau }{P_{\epsilon}})$.



The left eigenvectors are given by the row vectors of the inverse matrix $\mathcal{L}_{1} = \mathcal{R}_{1}^{-1}$

\begin{align*}
  \mathcal{R}_{1}^{-1} = \frac{1}{c^2}
  \left[
  \begin{array}{cccccc}
   \frac{1}{4} (2 c  v_{1}+\omega ) & \frac{1}{2} (-c- \phi_{1} ) & -\frac{1}{2} \phi_{2}
    & -\frac{1}{2} \phi_{3}  & \frac{P_{\epsilon} \tau }{2} & \frac{P_{D_{e}}}{2}
     \\
   -\frac{v_{2} \omega }{2} & \phi_{1} v_{2}  & c^2+\phi_{2} v_{2}  &
     \phi_{3} v_{2}  & -\phi_{2}  & -P_{D_{e}} v_{2}
     \\
   \frac{2 \chi_{1}  c^2+\alpha_{1}  \omega \tau^{-1} }{2 \chi_{1} } & -\frac{\phi_{1} \alpha_{1}  }{\chi_{1} \tau } &
     -\frac{\phi_{2} \alpha_{1}  }{\chi_{1} \tau } & -\frac{\phi_{3} \alpha_{1} }{\chi_{1} \tau } &
     \frac{P_{\epsilon} \alpha_{1} }{\chi_{1} } & \frac{P_{D_{e}} \left(\alpha_{1} -2 c^2\right)}{\tau \chi_{1} }
      \\
   -\frac{y_{e} P_{D_{e}} \omega }{\chi_{1} \tau } & \frac{2 y_{e} P_{D_{e}} \phi_{1} }{\chi_{1} \tau } & \frac{2 y_{e} P_{D_{e}}
     \phi_{2} }{\chi_{1} \tau} & \frac{2 y_{e} P_{D_{e}} \phi_{3} }{\chi_{1} \tau} & -\frac{2 y_{e}
     P_{D_{e}} P_{\epsilon} }{\chi_{1} } & \frac{2 P_{D_{e}} \left(c^2-y_{e} P_{D_{e}} \right)}{\tau \chi_{1} }
      \\
   -\frac{v_{3} \omega }{2} & \phi_{1} v_{3}  & \phi_{2} v_{3}   &
     c^2+\phi_{3} v_{3}  & -\phi_{3}  & -P_{D_{e}} v_{3}
      \\
   \frac{1}{4} (\omega -2 c  v_{1}) & \frac{1}{2} (c-\phi_{1} ) & -\frac{1}{2} \phi_{2}
       & -\frac{1}{2} \phi_{3}  & \frac{P_{\epsilon} \tau }{2} & \frac{P_{D_{e}}}{2}
     \\
  \end{array}
  \right]
\end{align*}

\begin{align*}
  \mathcal{R}_{2}^{-1} = \frac{1}{c^2}
  \left[
  \begin{array}{cccccc}
   \frac{1}{4} (2 c  v_{2}+\omega ) & -\frac{1}{2} \phi_{1} & -\frac{1}{2} (c+\phi_{2})
    & -\frac{1}{2} \phi_{3}  & \frac{P_{\epsilon} \tau }{2} & \frac{P_{D_{e}}}{2}
     \\
   -\frac{v_{2} \omega }{2} & c^2 + \phi_{1} v_{1}  & \phi_{2} v_{1}  &
     \phi_{3} v_{1}  & -\phi_{1}  & -P_{D_{e}} v_{1}
     \\
   \frac{2 \chi_{2}  c^2+\alpha_{2}  \omega \tau^{-1} }{2 \chi_{2} } & -\frac{\phi_{1} \alpha_{2}  }{\chi_{2} \tau } &
     -\frac{\phi_{2} \alpha_{2}  }{\chi_{2} \tau } & -\frac{\phi_{3} \alpha_{2} }{\chi_{2} \tau } &
     \frac{P_{\epsilon} \alpha_{2} }{\chi_{2} } & \frac{P_{D_{e}} \left(\alpha_{2} -2 c^2\right)}{\tau \chi_{2} }
      \\
   -\frac{y_{e} P_{D_{e}} \omega }{\chi_{2} \tau } & \frac{2 y_{e} P_{D_{e}} \phi_{1} }{\chi_{2} \tau } & \frac{2 y_{e} P_{D_{e}}
     \phi_{2} }{\chi_{2} \tau} & \frac{2 y_{e} P_{D_{e}} \phi_{3} }{\chi_{2} \tau} & -\frac{2 y_{e}
     P_{D_{e}} P_{\epsilon} }{\chi_{2} } & \frac{2 P_{D_{e}} \left(c^2-y_{e} P_{D_{e}} \right)}{\tau \chi_{2} }
      \\
   -\frac{v_{3} \omega }{2} & \phi_{1} v_{3}  & \phi_{2} v_{3}   &
     c^2+\phi_{3} v_{3}  & -\phi_{3}  & -P_{D_{e}} v_{3}
      \\
   \frac{1}{4} (\omega -2 c  v_{2}) & \frac{1}{2} \phi_{1}  & \frac{1}{2} (c-\phi_{2})
       & -\frac{1}{2} \phi_{3}  & \frac{P_{\epsilon} \tau }{2} & \frac{P_{D_{e}}}{2}
     \\
  \end{array}
  \right]
\end{align*}
where $\phi_{i} = P_{\epsilon}\,\tau\, v_{i}$,
$\omega = \tau\, (P_{\epsilon}\,(v^2 - 2\epsilon) - 2\,P_{\tau}\,\tau)$, and
$\alpha_{i} = 2 y_{e} P_{D_{e}} - \tau \chi_{i}$.

%\end{footnotesize}



\bibliography{ms}


\end{document}
