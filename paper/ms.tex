\documentclass[onecolumn]{aastex62}
\usepackage{times}
\usepackage{epsfig}
\usepackage{amsmath, amsthm, amssymb}
\usepackage{color}
\usepackage{microtype}
\usepackage{float}
\usepackage{stfloats}
\usepackage{natbib}
\usepackage{verbatim}
\usepackage{placeins}
\graphicspath{{./figures/}}
\usepackage{wrapfig}
\hypersetup{backref,breaklinks,colorlinks,citecolor=blue}
\usepackage[all]{hypcap}
\usepackage{definitions}
\usepackage{subfiles}

\bibliographystyle{apj}

\newcommand{\subdate}{\today}
\newcommand{\shortauth}{Barker et al.}
\newcommand{\slugcom}{Draft - \today}
\newcommand{\Pos}{\textnormal{\tiny\textsc{Pos}}}
\newcommand{\TVD}{\textnormal{\tiny\textsc{Tvd}}}
%\newcommand{\slugcom}{Submitted to ApJL on \subdate}
%\slugcomment{\slugcom}


\lefthead{\sc \footnotesize \slugcom \hfill \shortauth}
\righthead{\sc \footnotesize \slugcom \hfill \shortauth}
%}

\bibliographystyle{aasjournal}
\received{\today}
\shorttitle{Discontinuous Galerkin methods with a nuclear EOS}

\begin{document}

\title{\thornado-hydro: Generalizing discontinuous galerkin methods for a nuclear
equation of state for supernova hydrodynamics}
\author{Brandon Barker}
\affiliation{Department of Physics and Astronomy, University of Tennessee Knoxville, TN 37996}

\author{Eirik Endeve}
\affiliation{Department of Physics and Astronomy, University of Tennessee Knoxville, TN 37996}
\affiliation{Computer Science and Mathematics Division, Oak Ridge National Laboratory, TN 37831}
\affiliation{Joint Institute for Computational Sciences, Oak Ridge National Laboratory, TN 37831}

%\begin{abstract}
%\end{abstract}
\begin{abstract}
A problem of high importance in computational astrophysics is obtaining accurate
solutions to the Euler equations of hydrodynamics. We are interested in solving
the Euler equations in the context of core collapse supernovae (CCSNe).
The toolkit for high-order neutrino-radiation hydrodynamics (\thornado)
is being developed for CCSNe simulations and related problems utilizing a spatial discretization
based on the discontinuous Galerkin (DG) method.
Because the Euler equations form a hyperbolic set of partial differential equations,
they may be linearized to advection equations by the introduction of
chracteristic variables related to the eigen-decomposition.
This use of characteristic variables also increases the efficiency of the slope
limiting process. However, the addition of the nuclear equation of state equation
to the Euler equations makes the decompisition nontrivial.
We introduce the framework for the characteristic decomposition of the Euler equations
with the inclusion of the nuclear EOS terms and
present results from some initial tests.
\end{abstract}

\keywords{supernovae: general -- hydrodynamics -- convection -- turbulence -- equation of state -- methods: numerical -- stars: massive -- stars: evolution}


\section{Introduction}
\label{Intro}
\todo{Fill in introduction. Contextualize the work.}
The core collapse supernovae (CCSNe) explosion mechanism is fundamentally three-dimensional
in nature \citep[see e.g.,][]{blondin:2006, muller:2012, oconnor:2018b}. Alongside general
reletavistic gravity, complex nuclear equations of state (EOS), and
neutrino transport, hydrodynamics must be accurately modelled,
creating a very challenging problem. Hydrodynamic instabilities are critical
in aiding the explosion, with complex phenomena such as turbulence and
convection play key roles in the CCSN mechanism
\citep{murphy:2011, murphy:2013, couch:2013, couch:2015a, radice:2016, mabanta:2018}.
The accurate and efficient modelling of the supernova hydrodynamics is crucial if the
explosion is to be realistically modelled.
For reviews of the CCSN mechanism, see \citet{bethe:1990, janka:2007, janka:2012a, janka:2016, burrows:2013, hix:2014, muller:2016, couch:2017}.

\todo{Talk about \thornado, DG methods, Euler Equations, possibly reference other codes that
employ different methods and those methods' drawbacks.}

... organized as follows:

\section{Euler Equations of Gas Dynamics in Cartesian Coordinates}
\todo{Use this form of the Euler Equations?}
The non-relativistic Euler equations of gas dynamics
\citep[see, e.g.,][for details]{leveque:2002} in cartesian coordinates are given by
the equations of conservation of mass,
\beq
  \pderiv{\rho}{t} + \pderiv{}{x}(\rho u) = 0
  \label{eq:massConservation}
\eeq
conservation of momentum,
\beq
	\pderiv{(\rho u)}{t} + \pderiv{}{x}(\rho u^2 + p) = 0
  \label{eq:momentumConservation}
\eeq
conservation of energy,
\beq
  \pderiv{E}{t} + \pderiv{}{x}((E+p)u) = 0
  \label{eq:energyConservation}
\eeq
and conservation of \textcolor{red}{electron density}
\beq
  \pderiv{D}{t} + \pderiv{}{x}(D u) = 0
  \label{eq:electronConservation}
\eeq
where $\rho$ represents mass density, $u$ the fluid velocity, $p$ the fluid
pressure, $D=\rho y_e$ is the \textcolor{red}{what is this?},
and $E=\epsilon \rho +\frac{1}{2}\rho u^2$ the total specific energy (internal plus kinetic).
The system of equations is closed by the EOS which links the pressure to
the quantities evolved by Equations
\eqref{eq:massConservation}-\eqref{eq:electronConservation}. The inclusion of
Equation \eqref{eq:electronConservation} is because we require a nuclear EOS.
We may rewrite Equations \eqref{eq:massConservation}-\eqref{eq:electronConservation}
in a more convenient way:

\beq
  \pderiv{\mathbf{U}}{t}
  +\pderiv{\mathbf{F}(\mathbf{U})}{x}
  =0,
  \label{eq:conservation}
\eeq
where $\mathbf{U} =(\rho,\rho u,E, D)^{T}$ is the vector of conserved quantities
and $\mathbf{F}(\mathbf{U})=(\rho u,\rho u^{2}+p,(E+p)u, uD)^{T}$
is the flux vector.

\section{Numerical Implementation}
\label{DG}

\todo{Add intro paragraph} \\
\textcolor{red}{Subsections or no?}

\subsection{The Discontinuous Galerkin Methods}
In our solver we have chosen the discontinuous Galerkin (DG) method
for our spatial discretization. In this section we will briefly discuss our
implementation of the DG method, introducing notation and concepts. Recall that
we seek solutions to the Euler equations of hydrodynamics, constituting a
conservation law of the form

\beq
  \partial_{t} \mathbf{U} + \partial_{x} \mathbf{F}(\mathbf{U}) = 0
  \label{eq:ode}
\eeq
Where $\mathbf{U}$ is the evolved state vector and $\mathbf{F}(\mathbf{U})$ is
the flux. In order to solve Equation \eqref{eq:ode}, we divide the computational
domain $D\subset \mathbb{R}$ into a disjoint union $\mathcal{T}$ of open elements
$\bK$ such that $D = \cup_{\bK \in \cT}\bK$. Each element $\bK$ is a box in the
coordinates
\beq
  \bK=\{\,\vect{x} = (\xL,x_H) : x \in K \},
\eeq
We let the approximation space $\mathbb{V}^{k}$ for the DG method
be constructed from the tensor product of one-dimensional
polynomials of degree $k$. Note that functions in $\mathbb{V}^{k}$
can be discontinuous across element interfaces. The DG problem is then to find
$\mathbf{U}_h \in \mathbb{V}^{k}$ which approximates $\mathbf{U}$ in Equation
\eqref{eq:ode}, such that $\forall$ $\phi \in \mathbb{V}^{k}$ and $\bK \in \cT$

\beq
  \partial_{t} \int_{\bK}\mathbf{U}_h\, \phi\, dx +
  \int_{\bK}\partial_{x}\, \mathbf{F(\mathbf{U})}\, \phi\, dx = 0.
\label{eq:semidiscreteDG_almost}
\eeq
Integrating the second term by parts, this becomes
\beq
\begin{split}
  & \partial_{t} \int_{\bK}\mathbf{U_{h}}\, \phi\, dx +
   \widehat{\mathbf{F}}(\mathbf{U}_h)\, \phi^{-} \big|_{x_{H}} \\
  & + \widehat{\mathbf{F}}(\mathbf{U}_h)\, \phi^{+} \big|_{x_{L}} +
   \int_{\bK}\mathbf{F(\mathbf{U}_h)}\, \partial_{x}\phi\, dx = 0.
\label{eq:semidiscreteDG}
\end{split}
\eeq
In Eq.~\eqref{eq:semidiscreteDG}, $\widehat{\vect{F}}(\vect{U}_{h})$ is a
numerical flux approximating the flux on the boundary of $\bK$.
The numerical flux function is evaluated using values from
both boundaries of an element; i.e.,
\begin{equation}
  \widehat{\vect{F}}(\vect{U}_{h})=\vect{f}(\vect{U}_{h}(x^{-},\tilde{\bx}),\vect{U}_{h}(x^{+},\tilde{\bx})),
\end{equation}
where superscripts $-/+$
indicate that the function is evaluated to the immediate left/right of the
interface. We use the Harten-Lax-van Leer (HLL) flux \citep{harten:1983} or
the HLLC flux \citep{toro:1994,mignone:2005} for all the numerical
experiments presented in Section \ref{sec:tests}.

In each element $K$, we use a nodal representation of the conserved variables
$\mathbf{U}$:

\beq
\begin{split}
  \mathbf{U}(x,t) \approx\, & \mathbf{U}_h (x,t) =
  \sum_{i=1}^{N}\vect{U}_{i}(t)\,\ell_{i}(x),
  \quad\text{where}\quad \\
  & \ell_{i}(\eta)=
  \prod_{\substack{j=1\\j\ne i}}^{N}\f{\eta-\eta_{j}}{\eta_{i}-\eta_{j}}
  \label{eq:conservedNodalExpansion}
\end{split}
\eeq
are Lagrange polynomials defined on $I = \{ \eta : \eta \in (-0.5,0.5) \}$,
and are constructed to interpolate the node set
$S_{N}=\{\eta_{i}\}_{i=1}^{N}\subset I$. The spatial coordinate $x$ and the
reference coordinate $\eta$ are related by the
mapping $x(\eta)=\xL+(0.5+\eta)\,\Delta x$.
Then, for any $\eta_{j}\in S_{N}$, $\ell_{i}(\eta_{j})=\delta_{ij}$,
so that $\vect{U}_{h}(x(\eta_{j}),t)=\vect{U}_{j}(t)$.
First we define the $M$-point quadrature $Q_{M}:C^{0}(I)\to\mathbb{R}$
with abscissas $\hat{S}_{M}=\{\eta_{q}\}_{q=1}^{M}$ and weights
$\{w_{q}\}_{q=1}^{M}$, normalized such that $\sum_{q=1}^{M}w_{q}=1$ in order to
evaluate the integrals in Eq.~\eqref{eq:semidiscreteDG}..
We use the $M$-point Legendre-Gauss quadrature, which is exact for
polynomials of degree $\le 2M-1$.
Then, if $P_{h}(x)$ is such a polynomial, we have
\beq
  \f{1}{\Delta x}\int_{K}P_{h}(x)\,dx=\int_{I}P_{h}(\eta)\,d\eta=\sum_{q=1}^{M}w_{q}\,P_{h}(\eta_{q}).
\eeq

For the sake of efficiency we let $M=N$ and $S_{N}=\hat{S}_{N}$, which is a
spectral-type nodal collocation DG approximation \citep{bassi:2013}.
Inserting Eq.~\eqref{eq:conservedNodalExpansion} into Eq.~\eqref{eq:semidiscreteDG},
letting $\phi(x)=\ell_{k}(x)$, using the quadratures defined above,
we obtain\footnote{Are these summations?}
\begin{align}
  \pd{}{t}\int_{K}\vect{U}_{h}\,\phi\,dx
  &\approx w_{k}\,\pd{}{t}\vect{U}_{k}\,\Delta x
  \label{eq:timeDerivativeTerm}
\end{align}
for the time derivative piece.
Similarly, the last term on the left-hand side of Eq.~\eqref{eq:semidiscreteDG}
becomes
\beq
  \int_{K}\vect{F}(\vect{U}_{h})\,\pderiv{\phi}{x}\,dx
  \approx \sum_{q=1}^{N}w_{q}\,\vect{F}(\vect{U}_{q})\,\pderiv{\ell_{k}}{\eta}(\eta_{q}).
  \label{eq:volumeTerm}
\eeq
Now we may combine Eqs.~\eqref{eq:timeDerivativeTerm}-\eqref{eq:volumeTerm} in
Eq.~\eqref{eq:semidiscreteDG} resulting in the semi-discrete form
\beq
\begin{split}
  \deriv{\vect{U}_{k}}{t}
  & =-\f{1}{w_{k}\Delta x}
  \Big\{
  \Big[\,
    \widehat{\vect{F}}\ell_{k}\big|_{x_{H}}
     -\widehat{\vect{F}}\ell_{k}\big|_{\xL}
  \,\Big] \\
  & -\sum_{q=1}^{N}w_{q}\,\vect{F}(\vect{U}_{q})\,\pderiv{\ell_{k}}{\eta}(\eta_{q})
  \Big\}.
  \label{eq:semidiscreteDiscretized}
\end{split}
\eeq
We now have a system of ordinary
differential equations (ODEs), which may be evolved in time with an ODE solver.
In Section~\ref{sec:tests} we use the third-order strong
stability-preserving Runge-Kutta (SSP-RK3) method\footnote{Change later if necessary} \citep{shu:1988}.

\subsection{Slope Limiting}
A common feature of numerical ODEs is unphysical oscillations in the solutions.
It is therfore of great interest in the DG algorithm to implement slope limiting
of the polynomial $\mathbf{U}_h$. We use the total variation diminishing (TVD)
slope limiter \citep[see, e.g.,][]{cockburn:1998} in conjunction with the
troubled cell indicator discussed in \citet{fu:2017} to prevent excessive limiting.
The general $s$-stage Runge-Kutta time stepping algorithm, including
the limiting process, can then be summarized as in \citet{cockburn:2001}:
\begin{itemize}
  \item[1.] Set $\bar{\vect{U}}^{(0)} = \bar{\vect{U}}^{n}$,
  \item[2.] For $i=1,\ldots,s$ compute:
  \begin{equation}
  \small  \bar{\vect{U}}^{(i)}
    = \Lambda^{\Pos}\Big\{\Lambda^{\TVD}\Big\{\sum_{j=0}^{i-1}\alpha_{ij}\,\bar{\vect{U}}^{(j)}+\beta_{ij}\,\Delta t\,\bar{\vect{F}}\big(\bar{\vect{U}}^{(j)}\big)\Big\}\Big\},
    \label{eq:rkStages}
  \end{equation}
  \item[3.] Set $\bar{\vect{U}}^{n+1}=\bar{\vect{U}}^{(s)}$.
\end{itemize}
Above, the TVD limiter and the `positivity' limiter preventing unphysical
states are denoted by the operators $\Lambda^{\TVD}\{\}$
and $ \Lambda^{\Pos}\{\}$, respectively. The SSP-RK3 coefficients $\alpha_{ij}$
and $\beta_{ij}$ may be found in Table~2.1 in \citet{cockburn:2001}.

\subsection{Characteristic Decomposition}
Experience has shown that the slope limiting described in the previous section
is more effecient when performed on the so-called `characteristic variables`
as opposed to the conserved variables $\mathbf{U}_h$
\citep[see, e.g.,][for a description]{cockburn:1998}. We begin by
rewriting Eq.~\ref{eq:conservation} as follows
\beq
  \pderiv{\mathbf{U}}{t}
  + \pderiv{\mathbf{U}}{x} \pderiv{\mathbf{F(\mathbf{U})}}{\mathbf{U}}
  = 0.
  \label{eq:charEq}
\eeq
Because the Euler equations for a systemer of hyperblic
partial differential equations \citep[see, e.g.,][]{leveque:1992}, we can decompose the
Jacobian of the flux vector as

\beq
  \pderiv{\mathbf{F(\mathbf{U})}}{\mathbf{U}} =
  \mathcal{R} \Lambda \mathcal{R}^{-1},
\eeq
where the columns of $R$ contain the right eigenvectors of the Jacobian,
the rows of $\mathcal{R}^{-1}$ contain the left eigenvectors, and
$\Lambda$ is a diagonal matrix containing the eigenvalues of the Jacobian.
At this point, we will introduce the characteristic variable
$\mathbf{w} = \mathcal{R}^{-1}\mathbf{U}$. Multiplying both sides of Equation
\eqref{eq:charEq} by $\mathcal{R}^{-1}$, we are able to linearize the system of equations to
a system of advection equations

\beq
  \pderiv{\mathbf{w}}{t} +
  \Lambda \pderiv{\mathbf{w}}{x}
  = 0.
\eeq
Solutions to these advection equations are far simpler to obtain than the
previous equations. Limiting may then be applied to dampen oscillations in the
solutions for the characteristic variables $\mathbf{w}$, which are then
transformed back to the solution of the conserved variables $\mathbf{U}$
\citep[see e.g.,][for a description]{cockburn:1998, schaal:2015a}.

\section{Derivation of Jacobian and Eigenvalues for the 3D Nuclear Case}

We assume that $p = p(\tau, e, n)$, where $\tau = \frac{1}{\rho}$.
Let the vector of conserved variables be $\textbf{U} = \{\rho, m_1, m_2, m_3, E, D\}$,
where $m_i = \rho v_i$ and $D = \rho Y_{e}$.
The flux vector is $\textbf{F}(\textbf{U}) =
\{m_{1}, m_{1}^{2}\tau + p, m_{1}m_{2}\tau, m_{1}m_{3}\tau,
(E+p)m_{1}\tau, Dm_{1}\tau\}$. \\


%\begin{table*}[ht!]

\begin{align}
  \centering
	\pderiv{\mathbf{F}(\mathbf{U})}{\mathbf{U}}
	= \left[
		\begin{array}{cccccc}
			0 & 1 & 0 & 0 & 0 & 0 \\
			-v_{1}^{2} -p_{\tau}\tau^{2} - p_{\epsilon}\tau(\epsilon - \frac{v^{2}}{2}) & v_{1}(2-p_{\epsilon}\tau)  & -p_{\epsilon}v_{2}\tau & -p_{\epsilon}v_{3}\tau  & p_{\epsilon}\tau  & p_{n} \\
			-v_{1}v_{2} & v_2 & v_1 & 0 & 0 & 0 \\
			-v_{1}v_{3} & v_{3} & 0 & v_{1} & 0 & 0 \\
			v_{1}(-H - p_{\tau}\tau^{2} -p_{\epsilon}\tau(\epsilon - \frac{v^{2}}{2})) & H - p_{\epsilon}v_{1}^{2}\tau  & -p_{\epsilon}v_{1}v_{2}\tau & -p_{\epsilon}v_{1}v_{3}\tau  & v_{1}(1+p_{\epsilon}\tau) & v_{1} p_{n} \\
			-v_{1} Y_{e}/m_{B} & Y_{e}/m_{B} & 0 & 0 & 0 & v_{1} \\
		\end{array}
    \right]
\end{align}
%\caption{}
%\end{table*}


\noindent The eigenvalues of this Jacobian are given by

\begin{align}
\Lambda =
\begin{bmatrix}
  v_{1} - c_{s} & 0 & 0& 0& 0& 0\\
  0 & v_{1} & 0 & 0 & 0 & 0 \\
  0 & 0 & v_{1} & 0 & 0 & 0 \\
  0 & 0 & 0 & v_{1} & 0 & 0 \\
  0 & 0 & 0 & 0 & v_{1} & 0 \\
  0 & 0 & 0 & 0 & 0 & v_{1} + c_{s}
\end{bmatrix}
\end{align}
where $c_{s} = \sqrt{\tau (n p_{n} + p p_{\epsilon}\tau - p_{\tau}\tau)}$ is
the local sound speed. The right eigenvectors are then given by
\begin{align*}
  \mathcal{R}_{1} =
  \left[
  \begin{array}{cccccc}
   1 & 0 & 1 & 1 & 0 & 1 \\
   v_{1}-c & 0 & v_{1} & v_{1} & 0 & c+v_{1} \\
   v_{2} & 1 & 0 & 0 & 0 & v_{2} \\
   v_{3} & 0 & 0 & 0 & 1 & v_{3} \\
   h-c v_{1} & v_{2} & \beta & 0 & v_{3} & h+c v_{1} \\
   n \tau  & 0 & 0 & \frac{\tau  \chi }{2 p_{n}} & 0 & n \tau  \\
  \end{array}
  \right]
\end{align*}
where the follow definitions have been used:
$h_{n} = \frac{c^2}{p_{\epsilon}\tau} + k $, $k = \frac{-n p_{n}
+ p_{\epsilon} (\frac{1}{2}v^2 + \epsilon) + p_{\tau}\tau}{p_{\epsilon}}$,
$\delta_{1} = v_{1}^{2}-v_{2}^{2}-v_{3}^{2}$,
$\chi = p_{\epsilon} ( \delta_{1} + 2\epsilon) + 2p_{\tau}\tau$, and
$\beta = \frac{1}{2} (\delta_{1}+2 \epsilon +\frac{2 p_{\tau} \tau }{p_{\epsilon}})$.
\begin{align*}
  \mathcal{R}_{1}^{-1} = \frac{1}{c^2}
  \left[
  \begin{array}{cccccc}
   \frac{1}{4} (2 c  v_{1}+\omega ) & \frac{1}{2} (-c- \phi_{1} ) & -\frac{1}{2} \phi_{2}
    & -\frac{1}{2} \phi_{3}  & \frac{p_{\epsilon} \tau }{2} & \frac{p_{n}}{2}
     \\
   -\frac{v_{2} \omega }{2} & \phi_{1} v_{2}  & c^2+\phi_{2} v_{2}  &
     \phi_{2} v_{3}  & -\phi_{2}  & -p_{n} v_{2} \\
   \frac{2 \chi  c^2+\alpha  \omega }{2 \chi } & -\frac{\phi_{1} \alpha  }{\chi } &
     -\frac{\phi_{2} \alpha  }{\chi } & -\frac{\phi_{3} \alpha }{\chi } &
     \frac{p_{\epsilon} \alpha  \tau }{\chi } & \frac{p_{n} \left(\alpha  \tau -2 c^2\right)}{\tau  \chi } \\
   -\frac{n p_{n} \omega }{\chi } & \frac{2 n p_{n} \phi_{1} }{\chi } & \frac{2 n p_{n}
     \phi_{2} }{\chi } & \frac{2 n p_{n} \phi_{3} }{\chi } & -\frac{2 n
     p_{n} p_{\epsilon} \tau }{\chi } & \frac{2 p_{n} \left(c^2-n p_{n} \tau \right)}{\tau  \chi } \\
   -\frac{v_{3} \omega }{2} & \phi_{3} v_{1}  & \phi_{3} v_{2}   &
     c^2+\phi_{3} v_{3}  & -\phi_{3}  & -p_{n} v_{3} \\
   \frac{1}{4} (\omega -2 c  v_{1}) & \frac{1}{2} (c-\phi_{1} ) & -\frac{1}{2} \phi_{2}
       & -\frac{1}{2} \phi_{3}  & \frac{p_{\epsilon} \tau }{2} & \frac{p_{n}}{2}
     \\
  \end{array}
  \right]
\end{align*}
where $\phi_{i} = p_{\epsilon}\tau v_{i}$.

\todo{What is the physical interpretation of h and k? I tried to follow \citet{schaal:2015a},
and in their eigenvectors (Ideal), h is specific enthalpy and k is specific kinetic energy.}
The matrix of left eigenvectors is given by


\appendix

\section{Derivatives}


Derivatives of pressure with respect to $\tau$, e, and n. \\
\begin{footnotesize}
\begin{eqnarray}
	\frac{\partial{p}}{\partial{e}} &=& \left(\frac{\partial{e}}{\partial{T}}\right)^{-1}\left(\frac{\partial{p}}{\partial{T}}\right) \\
	\frac{\partial{p}}{\partial{n}} &=& \frac{m_B}{\rho} \left[ \frac{\partial{p}}{\partial{y_e}} -
          \frac{\partial{e}}{\partial{y_e}}\left(\frac{\partial{e}}{\partial{T}}\right)^{-1}\left(\frac{\partial{p}}{\partial{T}}\right)\right]\\
	\frac{\partial{p}}{\partial{\tau}} &=& -\tau^{-2} \left[ \frac{\partial{p}}{\partial{\rho}} - \frac{y_e}{\rho} \left( \frac{\partial{p}}{\partial{y_e}} -
          \frac{\partial{e}}{\partial{y_e}} \left(\frac{\partial{e}}{\partial{T}}\right)^{-1}\left(\frac{\partial{p}}{\partial{T}}\right)\right) -
          \frac{\partial{e}}{\partial{\rho}}\left(\frac{\partial{e}}{\partial{T}}\right)^{-1}\left(\frac{\partial{p}}{\partial{T}}\right)\right]
\end{eqnarray}
\end{footnotesize}

\acknowledgements
We thank people.

\bibliography{ms}


\end{document}
